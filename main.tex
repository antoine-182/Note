\documentclass[11pt,a4paper,twoside,openright]{article}
% \documentclass[14pt,a4paper,twoside,openright]{report}
\usepackage{lmodern}
\usepackage[utf8]{inputenc}
\usepackage[french]{babel}
\usepackage[T1]{fontenc}
\usepackage{amsmath}
\usepackage{amsfonts}
\usepackage{amssymb}
\usepackage{graphicx}
\usepackage[hidelinks]{hyperref}
\usepackage[top=2.5cm, bottom=4.5cm, left=1.5cm, right=1.5cm]{geometry}
% \usepackage[top=0cm, bottom=0cm, left=2cm, right=2cm]{geometry}
\usepackage{fancyhdr}
\usepackage[perpage, bottom]{footmisc} % footnote below figures
\usepackage{verbatim}
\usepackage{enumitem} % pimp les itemizes
%Pour mettre deux figures côtes à côtes
\usepackage{caption} 
\usepackage{subcaption}
% reference figure
% \usepackage{hyperref}
%cropping figures
\usepackage{adjustbox}
\usepackage{setspace} %spacing
\usepackage{ulem} % \ul
% \usepackage{algorithm2e} % pseudo code
% \usepackage{siunitx}
\usepackage{multirow} % pour des tableaux multicolumns
% \usepackage{float} 



\usepackage{chngcntr}
\counterwithin{figure}{section}
\counterwithin{equation}{section}


\usepackage[titletoc,page]{appendix}
% \usepackage{chngcntr}
% \counterwithout{figure}{section}

% \setlength\parindent{0pt} % Removes all indentation from paragraphs

\usepackage{xcolor}
\definecolor{Mgray}{RGB}{240,240,240}
\usepackage{listings} % for code
\usepackage{courier} % change the font
\lstset{
    basicstyle=\footnotesize\ttfamily, % Default font
    numbers=left,              % Location of line numbers
    numberstyle=\tiny,          % Style of line numbers
    % stepnumber=2,              % Margin between line numbers
    numbersep=5pt,              % Margin between line numbers and text
    tabsize=2,                  % Size of tabs
    extendedchars=true,
    breaklines=true,            % Lines will be wrapped
    keywordstyle=\color{red},
    frame=lrtb,
    stringstyle=\color{white}\ttfamily, % Color of strings
    showspaces=false,
    showtabs=false,
    xleftmargin=17pt,
    framextopmargin = 3pt,
    framexleftmargin=17pt,
    framexrightmargin=5pt,
    framexbottommargin=4pt,
    %backgroundcolor=\color{lightgray},
    showstringspaces=false
}

\def\enstaLangPRE{Projet de Fin de Master}
\def\enstaLangSpecialite{Spécialité :}
\def\enstaLangAnneeScolaire{Année Scolaire :}
\def\enstaLangMentionConfidentialite{Mention de non-confidentialité}
\def\enstaLangAuteur{Auteur :}
\def\enstaLangTuteurENSTA{Tuteur ENSTA Paris :}
\def\enstaLangPromo{Promotion :}
\def\enstaLangTuteurOrganisme{Tuteurs organisme d'accueil :}
\def\enstaLangStageEffectueDu{Stage effectué du}
\def\enstaLangStageEffectueAu{au}
\def\enstaLangNomOrganisme{Nom de l'organisme d'accueil :}
\def\enstaLangAdresse{Adresse :}
%%%%%%%%%%%%%%%%%%%%%%%%%%%%%%%%%
%%          VARIABLES          %%
%%%%%%%%%%%%%%%%%%%%%%%%%%%%%%%%%
\newcommand{\logo}[1]{\def\enstaLogoCorp{#1}}
\newcommand{\specialite}[1]{\def\enstaSpecialite{#1}}
\newcommand{\annees}[1]{\def\enstaAnneeScol{#1}}
\newcommand{\titre}[1]{\def\enstaTitre{#1}}
\newcommand{\soustitre}[1]{\def\enstaSousTitre{#1}}
\newcommand{\illustration}[1]{\def\enstaIllustration{#1}}
\newcommand{\confidentialite}[1]{\def\enstaConfidentialite{#1}}
\newcommand{\auteur}[1]{\def\enstaAuteur{#1}}
\newcommand{\promotion}[1]{\def\enstaPromotion{#1}}
\newcommand{\tuteurENSTA}[1]{\def\enstaTuteurENSTA{#1}}
\newcommand{\tuteurOrganisme}[1]{\def\enstaTuteurOrganisme{#1}}
\newcommand{\dateDebut}[1]{\def\enstaDateDebut{#1}}
\newcommand{\dateFin}[1]{\def\enstaDateFin{#1}}
\newcommand{\organisme}[1]{\def\enstaOrganisme{#1}}
\newcommand{\adresseOrganisme}[1]{\def\enstaAdresseOrganisme{#1}}

\logo{figure/obs.jpg}
\annees{2019}
\titre{Etude des processus thermiques dans les régions de photodissociation (PDR)}
\soustitre{}
%\illustration{atlas_0.jpeg}
\confidentialite{Ce document est non confidentiel. Il peut donc
être communiqué à l’extérieur sous format papier mais également diffusé sous
format électronique}
\auteur{Antoine NASSER}
\promotion{2019}
\tuteurENSTA{Jérôme PEREZ}
\tuteurOrganisme{Franck Le Petit \\ Emeric Bron}
\dateDebut{18/03/2019}
\dateFin{17/09/2019}
\organisme{Observatoire de Paris}
\adresseOrganisme{61 AVENUE DE L'OBSERVATOIRE\\ 75014  Paris \\France}


%%%%%%%%%%%%%%%%%%%%%%%%%%%%%%%%%
%%        HEADER/FOOTER        %%
%%%%%%%%%%%%%%%%%%%%%%%%%%%%%%%%%

% redefine a style, here it is plain
% \fancypagestyle{fancy}{ 
% \fancyhf{}
% \fancyfoot[RO, LE] {\thepage}
% \fancyhead[RO] {\leftmark}
% \fancyhead[LE] {\rightmark}
% }

\pagestyle{fancy}
\renewcommand{\sectionmark}[1]{\markright{#1}}
\fancyhf{}
\fancyhead[CO]{\nouppercase{\rightmark}}
\fancyhead[CE]{\enstaTitre}
\fancyfoot[CO, CE]{\thepage}

% redefine a style, here it is plain
\fancypagestyle{nofancy}{ 
\fancyhf{}
\fancyhead[CO,CE]{}
\fancyfoot[CO, CE]{\thepage}
}

% appendix
\fancypagestyle{appendix}{ 
\fancyhf{}
\fancyhead[CO,CE]{\nouppercase{\rightmark}}
\fancyfoot[CO, CE]{\thepage}
}

% %%%%%%%%%%%%%%%%%%%%%%%%%%%%%%%%%
% %%         FIRST  PAGE         %%
% %%%%%%%%%%%%%%%%%%%%%%%%%%%%%%%%%
\newcommand{\couverture}{
  \begin{titlepage}
  \bfseries
    \begin{center}
    \vspace*{\stretch{2}}
    \begin{tabular*}{\textwidth}{@{\extracolsep{\fill}} lr}
      \includegraphics[height=5cm]{figure/ensta.png} & \includegraphics[height=5cm]{\enstaLogoCorp}
      %\includegraphics[scale=0.4]{ensta.jpg} & \includegraphics[scale=0.4]{\enstaLogoCorp}
    \end{tabular*} \\
    \vspace*{\stretch{0.5}}
    \huge \enstaLangPRE \\
    \vspace*{0.3 cm}
    \enstaLangAnneeScolaire ~\enstaAnneeScol \\
    \vspace*{0.5 cm}
	\begin{center}
	\line(1,0){500}
	\end{center}
    \Huge \enstaTitre 
	\begin{center}
	\line(1,0){500}
	\end{center}
  %  \Large \enstaSousTitre \\
    \vspace*{\stretch{0.5}}
  %  \includegraphics[height=4cm]{\enstaIllustration} \\
    \vspace*{1 cm}
    {\color{red} 
    \normalsize \enstaConfidentialite} \\
    \vspace*{\stretch{0.5}}
    \vspace{1cm}
    \begin{minipage}{0.49\textwidth}
      \begin{center}
  \large \enstaLangAuteur \\
  \enstaAuteur \\
  \vspace*{0.8cm}
  \normalsize \enstaLangTuteurENSTA\\
  \enstaTuteurENSTA \\
      \end{center}
    \end{minipage}
    \begin{minipage}{0.49\textwidth}
      \begin{center}
  \large \enstaLangPromo \\
  \enstaPromotion \\
  \vspace*{0.5cm}
  \normalsize \enstaLangTuteurOrganisme \\
  \enstaTuteurOrganisme \\
      \end{center}
    \end{minipage} \\
    \vspace{1cm}
    \vspace*{\stretch{1}}
    \normalsize \enstaLangStageEffectueDu ~\enstaDateDebut ~\enstaLangStageEffectueAu ~\enstaDateFin \m\
    \vspace*{\stretch{0.5}}
    \normalsize
    \begin{tabular}{l}
      \begin{tabular}{l@{}l}
      \enstaLangNomOrganisme & \begin{tabular}{l}\enstaOrganisme\end{tabular}
      \end{tabular} \\
      \begin{tabular}{l@{}l}
      \enstaLangAdresse & \begin{tabular}[t]{l}\enstaAdresseOrganisme\end{tabular}
      \end{tabular}
    \end{tabular}
    \vspace*{\stretch{2}}
    \end{center}
  \end{titlepage}
  %\newgeometry{top=2.5cm, bottom=2.5cm, left=2.5cm, right=2.5cm}
}
% %%%%%%%%%%%%%%%%%%%%%%%%%%%%%%%%%
% %%   CONFIDENTIALITY  NOTICE   %%
% %%%%%%%%%%%%%%%%%%%%%%%%%%%%%%%%%
\newcommand{\pageConfidentialite}[1]{\partb{\enstaLangMentionConfidentialite}#1}


\newcommand{\xvec}[1]{\overrightarrow{#1}}
\newcommand{\zer}{\vspace{1em}}
\newcommand{\uncinq}{\texttt{1.5.2\_rev\_1900}}
\newcommand{\unsept}{\texttt{1.7.0\_stab\_1926}}

% \title{Note de Stage\\Observatoire de Meudon} 
% \author{Antoine Nasser}
% \date{Printemps Été (chaud) 2019}

%%%%%%%%%%%%%%%%%%%%%%%%%%%
%%        Chapitre        %%
%%%%%%%%%%%%%%%%%%%%%%%%%%
% \renewcommand\partname{}
% \renewcommand{\thepart}{\Roman{part}) }
\renewcommand{\thesection}{\Roman{section}}
\renewcommand{\thesubsection}{\Roman{section}.\arabic{subsection} }
\renewcommand{\thesubsubsection}{\Roman{section}.\arabic{subsection}.\arabic{subsubsection}}

%%%%%%%%%%%%%%%%%%%%%%
%%        GO        %%
%%%%%%%%%%%%%%%%%%%%%%

\begin{document}

\begin{spacing}{0.98}


% \maketitle
% \begin{center}{\Large Etude des processus de chauffage et de refroidissement \\ dans les régions de photodissociation (PDR)}\end{center}


% \setcounter{secnumdepth}{4}
% \vfill
% \begin{figure*}[!hb]
%         \centering 
%         \includegraphics[trim = {0 0 0 0cm},clip,width=0.8\textwidth]{figure/LERMA2.jpg}
% \end{figure*}
% \vfill

\couverture

\clearpage
\thispagestyle{nofancy}
Ce document est non confidentiel. Il peut donc
être communiqué à l’extérieur sous format papier mais également diffusé sous
format électronique.

\clearpage 
\thispagestyle{nofancy}
\section*{Remerciements}

Je tiens à remercier tous ceux qui ont contribué, d’une manière ou d’une autre, à la réussite de ce projet. \newline 

Tout d’abord, je remercie Franck Le Petit et Emeric Bron, du LERMA, pour leur patience et leur soutien sans faille. J’ai eu énormément de plaisir à travailler avec eux. Ils m’ont tant appris aussi bien sur la physique des milieux interstellaires que sur le savoir-être du chercheur. \newline

Merci à Jacques le Bourlot, du LERMA, avec qui j’ai discuté pour la première fois du sujet et qui m’a accepté avec Franck Le Petit. Les discussions que l’on a eu ont toujours été très enrichissantes et agréables. \newline

Je remercie Laurent Verstraete, de l’IAS, auprès de qui j’ai beaucoup appris sur les poussières interstellaires.\newline

Je remercie mon enseignant référent de l’ENSTA, Jérome Perez qui a accepté d’évaluer mes travaux avec beaucoup de gentillesse. \newline

Je remercie par ailleurs toute l’équipe de Meudon pour son accueil chaleureux.\newline

Je remercie également mes co-stagiaires, Raphael Meshaka et Flavien Barbier, avec qui j’ai partagé bien plus qu'un bureau.  \newline

Je pense enfin à ma famille qui me soutient et m’encourage dans tout ce que j’entreprends. 


\newpage 

\clearpage 
\thispagestyle{nofancy}
\section*{Résumé}

A l'aide du code PDR de Meudon, nous avons revisité les processus de chauffages qui se déroulent au sein des PDR. L'ajout du chlore dans le réseau chimique à révélé l'existence d'une bistabilité thermique dans les bords atomiques des PDR denses et fortement illuminées. Néanmoins, nous n'avons pas de traceurs atomiques permettant de déceler son existence. La mise à jour du réseau chimique du code PDR a ouvert des nouvelles voies de formations du traceur $\mathrm{CO}$ qui est amorcée par la réaction  $\mathrm{O} + \mathrm{H}_2 \rightarrow \mathrm{OH} + \mathrm{H}$. L'utilisation de nouvelles prescriptions pour calculer les taux de destruction collisionnelle du $\mathrm{H}_2$ ont montré que les réactions chimiques pouvaient chauffer les PDR denses à l'entrée des nuages moléculaires ce qui a un impact fort sur le diagramme d'intensité des raies du $\mathrm{CO}$. Enfin le nouveau calcul des taux des transitions par collisions du $\mathrm{H}_2$ ont révélé l'existence d'une nouvelle instabilité thermique en fin de zone atomique pour les PDR fortement illuminées. Par manque de temps, son origine reste inconnue.

%%%%%%%%%%%%%%%%%%%%%%%%%%%%%%%%%%%%%%%%%%%%%%%%%%%%%%%%%%%%%%%%%%%%%%%%%%%%%%%%%%

\clearpage 
\setcounter{secnumdepth}{4}
\thispagestyle{nofancy}
\tableofcontents

\setcounter{figure}{0}    

\newpage



\begin{figure}[!p]
    \centering
    \includegraphics[trim = {0 0 0 0},clip, width=0.6\textwidth]{figure/omega.pdf}
    \vspace{1em}
    \caption{M17 ou nébuleuse du Cygne.
    % http://hubblesite.org/image/1331/news_release/2003-13
    Crédits NASA/ESA/J. Hester (ASU)}
    \vspace{1em}
    \begin{minipage}{\textwidth}
    Le rouge représente l'émission du soufre, le vert l'hydrogène et le bleu l'oxygène. Le nuage moléculaire froid est creusé et chauffé par le champ de rayonnement UV intense émis par un amas de jeunes étoiles qui serait situé hors du cadre en haut à gauche.
    \end{minipage} 
    \label{fig:m17}
\end{figure}{}


\begin{figure}[!p]
    \centering
    \includegraphics[width = 0.6\textwidth]{figure/mystic.pdf}
    \caption{Nébuleuse de la Carène, \og Montagne mystique \fg{}. Image composite des émissions de l’oxygène (en bleu), hydrogène et azote (vert) et soufre (rouge). Crédits HST/NASA.}
    %https://www.jpl.nasa.gov/spaceimages/details.php?id=PIA15985
    \vspace{1em}
    \begin{minipage}{\textwidth}
    Cette image prise par Hubble montre la forme complexe du milieu interstellaire. Les filament bleus-verts révèlent des jeunes étoiles brillantes situés hors cadre et à des années lumière de la structure qui érodent les colonnes de gaz par leurs champ UV intense. Des milliers d'étoiles sont enfouies dans le nuage interstellaire qui absorbent leurs émissions et nous empêche de les détecter. Aux sommets des colonnes, on voit les signes d'étoiles récemment formées qui éjectent de la matière dans des jets chauds tourbillonnant venant s'accréter sur la surface de l'étoile.
    \end{minipage}
    \label{fig:intro:mystic}
\end{figure}
 
 

\newpage
\setcounter{secnumdepth}{-1}
\part{Introduction}


Le contexte général de cette étude est le gaz et des poussières se trouvant entre les étoiles, il s'agit du milieu interstellaire (le MIS). Les poussières représentent à peine $1\%$ de la masse du milieu interstellaire. Les $99\%$ de gaz est constitué, en masse, de $\sim 74\%$ d'hydrogène, $\sim 25\%$ d'hélium et $<1\%$ d'éléments dit \og lourds \fg{} soit des éléments ayant un numéro atomique $Z$ supérieur à celui de l'hélium (carbone, oxygène, azote...).\newline 

% Actuellement, on sait que la masse de la majorité des galaxies est sous forme de matière noire et que les baryons représentent $10\%$ de leurs masses et sont responsables de leurs aspects visibles. Dans notre Voie Lactée, la masse baryonique est principalement sous forme d'étoiles et seulement $10\%$ est trouvé dans le milieu interstellaire qui est distribué principalement autour du plan galactique. Les poussières représentent à peine $1\%$ de la masse du milieu interstellaire. Les $99\%$ de gaz est constitué, en masse, à $\sim 74\%$ d'hydrogène, $\sim 15\%$ d'hélium et $<1\%$ d'éléments dit \og lourds \fg{} soit des éléments ayant un numéro atomique $Z$ supérieur à celui de l'hélium (carbone, oxygène, azote...).\newline 

Le milieu interstellaire est intimement lié à la formation des étoiles. Celles-ci se forment dans des nuages moléculaires denses et froids qui sont confinés par leur propre gravité. L'effondrement d'un nuage moléculaire sur lui-même amorce la formation d'étoiles qui, une fois allumées, émettent un champ de rayonnement UV intense. Ces nouvelles sources d'énergies interagissent à leur tour avec leur nuage parent en dissociant les molécules et en chauffant le gaz. La photoévaporation par les UV entraîne une expansion rapide du gaz et peut causer la dissipation du nuage parent, mais aussi, comprimer localement le nuage ce qui peut amener à la formation de nouvelles étoiles (figure \ref{fig:m17}). On appelle parfois ces régions de formation des \og pouponnières d'étoiles \fg{} tant leurs naissances amorcent la création, en cascade, d'autres étoiles (figure \ref{fig:intro:mystic}). Ainsi, comprendre le feedback radiatif des jeunes étoiles sur leurs nuages parents permet de mieux comprendre la formation des étoiles. C'est également une problématique importante en recherche extragalactique car elle contribue à déterminer les taux de formation et de destruction des étoiles dans les galaxies. \newline 

% Pour la présentation
% https://www.britannica.com/place/Orion-Nebula

Ces jeunes étoiles interagissent avec les nuages moléculaires dans des interfaces que l'on appelle des régions photon-dominées ou régions de photo-dissociations (PDR). Il s'agit de région du milieu interstellaire où la chimie et la température du gaz sont fortement influencées par le champ UV émis. Les observations Herschel\footnote{Téléscope spatial développé par l'\textit{European Spatial Agency} et en activité de 2009 à 2013. Il avait pour missions d'étudier dans les bandes de l'infrarouge lointain et submillimétriques la formation des galaxies et la naissance des étoiles. \\ 
 http://sci.esa.int/herschel/} et ALMA\footnote{
L'\textit{Atacama Large Millimeter/submilimmeter Array} (ALMA) est un réseau de télescope (plus d'une soixantaine) installé sur le plateau Chajnantor au Chilie. \\
 https://www.eso.org/public/france/teles-instr/alma/} ont sondé la structure de ces régions et révolutionné notre compréhension des mécanismes d'interactions \cite{Goicoechea2016, COJoblin}.
Proche de l'étoile, le champ UV photionise entièrement le gaz. Les photons ayant une longueur d'onde inférieure à la limite de Lyman ($\lambda \leq 912 \AA$ soit $h\nu \geq 13.6$ eV) ionisent l'hydrogène et les espèces ayant un potentiel de ionisation (PI) supérieur à celui de l'hydrogène, c'est la \og région $\mathrm{HII}$ \fg{} \footnote{Ou \og région $\mathrm{H}^+$ \fg{}. $\mathrm{HII}$ fait référence à $\mathrm{H}^+$, le II signifie qu'il a été ionisé une fois.}. Une PDR commence à partir de la zone où les photons émis par les étoiles ne peuvent plus photoioniser les atomes d'hydrogène (photons dans l'UV lointain, $6<h\nu<13.6$ eV) mais seulement dissocier des molécules ($\mathrm{H}_2 + h\nu \rightarrow \mathrm{H}+\mathrm{H}$) ou ioniser des espèces ayant un PI inférieur à $13.6$ eV comme le carbone. A mesure que l'on s'enfonce dans le nuage et que l'on s'éloigne de l'étoile, les photons UV sont progressivement absorbés par les poussières (ou grains) et le gaz comme $\mathrm{H}_2$ ou $\mathrm{C}$.
Le gaz est principalement sous forme atomique ($\mathrm{H}$) et on nomme cette région la zone atomique que l'on appelle aussi \og région \mathrm{HI} \fg{}. Il vient une profondeur dans le nuage où le flux de photon UV capable de dissocier les molécules $\mathrm{H}_2$ est totalement absorbé permettant au gaz de devenir moléculaire ($\mathrm{H}_2$). La présence de $\mathrm{H}_2$ amorce une chimie chaude qui initie la formation et l'excitation de molécules par exemple le $\mathrm{CO}$ dont on observe les raies d'émissions \cite{COJoblin}. On utilise les émissions de molécules comme des traceurs des processus physico-chimiques qui se déroule dans la PDR. La figure \ref{fig:intro:struct} schématise la structure d'une PDR. Ainsi, le flux UV des jeunes étoiles assurent la formation des espèces moléculaires au sein du nuage et l'excitation des raies observées. Pour interpréter les raies d'émissions, les astronomes utilisent des modèles PDR qui calculent le transfert radiatif à travers le nuage, la chimie et le bilan thermique. Les modèles permettent de comprendre la composition chimique des nuages et leurs températures qui sont déterminées à partir des processus de chauffage et de refroidissement. \newline 

\begin{figure}[!h]
    \centering
    \includegraphics[width = 0.8\textwidth]{figure/structurepdr.pdf}
    \caption{Structure d'une région de photo-dissociation. Les photons UV émis par les étoiles viennent de la gauche. Image tirée de \cite{Goicoechea2016}. $N_\mathrm{H}$ en $\mathrm{cm}^{-2}$ est la colonne densité de l'hydrogène. Elle correspond à la quantité de matière entre le bord gauche et la position dans le nuage le long de la ligne de visée.}
    \label{fig:intro:struct}
\end{figure}

On a découvert l'existence des phénomènes de chauffage et de refroidissement dans le milieu interstellaire à partir des années 60. Et jusqu'aux années 90, des formules analytiques permettant d'estimer ces processus ont été proposées. Elles se fondait sur notre connaissance des quantités physiques et des populations de grains que l'on avait à cette époque. Aujourd'hui notre compréhension des PDR est beaucoup plus fine notamment sur les poussières interstellaires grâce aux observations Planck\footnote{Planck est un satellite (ESA, en activé de 2009 à 2012) ayant pour mission de cartographier le fond diffus cosmologique qui est le rayonnement de l'Univers primordial. Il trouve son origine dans le rayonnement émis directement par les étoiles et celui des poussières. \\
https://www.cosmos.esa.int/web/planck} \cite{Jones_2017}. Pourtant nos modèles utilisent encore ces prescriptions alors que l'interprétation des observations en dépend directement. L'objectif du stage est donc de revoir l'ensemble des processus thermiques qui se déroulent au sein des PDR en s'intéressant à des processus de chauffage qui peuvent dominer localement dans des zones du nuage et jouer un rôle clé dans la compréhension de certains traceurs. Nous allons présenter les principaux phénomènes. Tout d'abord, l'effet photoélectrique sur les grains est le processus de chauffage dominant dans les PDR. L'absorption d'un photon UV par un grain entraîne l'éjection d'un électron de sa surface avec une énergie de l'ordre de l'électron volt. Ces électrons supra-thermiques (1 eV $\approx 12\,000$ K) entrent en collisions avec les particules de gaz ce qui thermalise les électrons et chauffe le gaz. Un second processus de chauffage majeur est la désexcitation collisionnelle du $\mathrm{H}_2$ : lorsque la molécule absorbe un photon, elle passe dans un état excité et peut se désexciter en entrant en collision avec les particules du gaz. Ainsi l'énergie du photon est convertie en agitation thermique ce qui chauffe le gaz. Cet effet peut être accentué par le pompage UV de la molécule $\mathrm{H}_2$ qui permet à plus de molécule de passer dans ses états excités. Ce processus sera abordé dans la seconde section. La chimie a également un impact sur la température du nuage. En effet, une réaction chimique est endothermique ou exothermique selon que l’enthalpie des produits est plus grande que celles des réactifs. Ainsi, une réaction exothermique redistribue son excédent d'énergie au gaz, ce qui le chauffe, et inversement une réaction endothermique prélève de l'énergie. Cependant, les PDR se refroidissent principalement par les émissions du gaz. Cela concerne la molécule $\mathrm{H}_2$ mais aussi les ions ($\mathrm{C}^+$, $\mathrm{S}^+$ ...), les atomes ($\mathrm{O}$, $\mathrm{N}$, $\mathrm{C}$ ...) et les autres molécules ($\mathrm{OH}^+$ ...) du nuage. Lorsque une espèce passe dans un état excité par une collision, elle se désexcite par émission spontanée, en émettant un photon dans une direction aléatoire. L'énergie d'agitation du gaz est convertie en photon et est perdue. Il y enfin le couplage gaz-grains qui a tendance à refroidir le gaz : les grains généralement plus froids entrent en collisions avec les particules du gaz et se thermalisent. Il existe également d'autres phénomènes dont l'on ne parlera pas ici comme le chauffage par les rayons cosmiques. \newline 

L'étude est menée à l'aide du code PDR de Meudon \cite{LePetit2006} qui est un modèle complet de PDR. C'est un modèle stationnaire qui est correct dans la limite où les photoréactions sont très rapides devant l'évolution dynamique du nuage, et le champs de rayonnement reste constant. Le code PDR résout dans une géométrie simplifié 1D le transfert du rayonnement à travers le gaz, la chimie et le bilan thermique. Comme nous l'avons vu, ces trois éléments sont couplés et il faut les résoudre de manière cohérente. Le transfert de rayonnement est déterminé en résolvant l'équation de transfert par une méthode spectrale. La chimie fait intervenir des centaines d'espèces qui sont reliés par des milliers de réactions. Les densités solutions sont calculées par une méthode du point fixe (Newton-Raphson). Le bilan thermique est résolu par dichotomie qui est une méthode simple mais robuste. Le code découpe en cellule la PDR et calcule, en parcourant chaque cellule, le transfert radiatif, les densités des espèces du gaz et la température d'équilibre. Comme l'état d'une cellule affecte le transfert de rayonnement des cellules voisines, le code itère à travers la PDR jusqu'à ce qu'il converge vers une solution stable. Ainsi le code PDR obtient des profils de densité et de température en fonction de la profondeur du nuage. Il produit également des diagrammes d'intensité de raies qui sont comparés aux observations. Enfin, les paramètres que l'on introduit pour un modèle de PDR sont l'intensité $\chi$ du rayonnement en unité d'Habing \footnote{Est parfois utilisé $G_0 = 1.71\chi$ en unité de Draine}, ainsi que sa densité $n_\mathrm{H}$ ou bien sa pression thermale $P$ selon que l'on considère un nuage à densité constante\footnote{On note la densité du nuage $n_\mathrm{H}$ qui est la densité des éléments hydrogène (sous toutes ses formes) et non pas la densité totale du nuage. Ainsi selon que le nuage est sous forme moléculaire ou atomique, la densité $n_\mathrm{H}$ reste bien constante.
\begin{equation*}
n_\mathrm{H} =  n(\mathrm{H}^+) + n(\mathrm{H}) + 2n(\mathrm{H}_2) +  ... \neq n_\mathrm{tot}
\end{equation*}
} ou à pression constante. Si des raies obtenues par calculs correspondent aux observations, alors on considère que les modèles dont ils sont issues décrivent bien la PDR.  \newline 

Au cours du stage, j'ai travaillé avec le code PDR. J'ai découvert que l'ajout de chlore dans le réseau chimique pouvait provoquer une instabilité thermique dans les bords atomique de nuage. Je me suis ensuite intéressé au chauffage par pompage UV de la molécule $\mathrm{H}_2$. J'ai d'abord testé l'impact sur les raies d'un nouveau calcul des taux de réactions impliquant le $\mathrm{H}_2$. Puis, j'ai analysé les changements qu'entraîne l'utilisation d'une nouvelle prescription pour calculer des taux de destructions de $\mathrm{H}_2$. J'ai enfin entamé une étude de l'impact sur les PDR des nouveaux taux de collisions de $\mathrm{H}_2$. Cette étude est malheureusement inachevée.



%%%%%%%%%%%%%%%%%%%%%%%%%%%%%%%%%%%%%%%%%%%%%%%%%%%%%%%%%%%%%%%%%%%%%%%%%%%%%%%%%%



\setcounter{secnumdepth}{4}
\clearpage



\section{Amplification de l'effet photoelectrique par le chlore}

\subsection{Neufeld and Wolfire}

Les réseaux d'astrochimie de PDR ont tendance à négliger le chlore qui a une abondance mineure dans les nuages (voir \autoref{tab:gaz}). Or des espèces dérivées du chlore ont été observées en absorption dans des nuages diffus ($n \sim 10^2\ cm^{-3}$). Des raies de $\mathrm{H}_2\mathrm{Cl}^+$,HCl et $\mathrm{HCl}^+$ excitées ont également été observées par Herschel dans plusieurs sources de la Galaxie et notamment dans le nuage moléculaire de la barre d'Orion \cite{Neufeld2012,Neufire2009}. J'ai ajouté le réseau chimique du chlore dans le modèle PDR afin d'interpréter ces résultats. \newline

\subsection{Analyse du rôle du chlore}

\begin{figure}[htbp]
    \centering
    \begin{subfigure}[t]{0.45\textwidth} % "0.45" donne ici la largeur de l'image
        \centering \includegraphics[trim = {0 0 0 1cm},clip,width=1\textwidth]{figure/model_Cl/PDR155_n_d1e5r1e4A2e1.png}
        \caption{Profil de température et densité en fonction de la profondeur dans le nuage}\label{fig:ClT}
    \end{subfigure}
    ~ 
    \begin{subfigure}[t]{0.45\textwidth}
        \centering \includegraphics[trim = {0 0 0 1cm},clip,width=1\textwidth]{figure/model_Cl/tb_PDR155_n_d1e5r1e4A2e1.png}
        \caption{Taux de chauffage et refroidissement en fonction de la température du gaz au bord atomique du nuage ($A_{\mathrm{V}} = 10^{-6}$)}\label{fig:ClHC}
    \end{subfigure}
    \caption{Profil de densité de l'hydrogène et de la température en fonction de l'extinction dans le visible.}
\end{figure}

J'ai étudié les réactions chimiques principales qui se déroulent en bord de nuage atomique et démontré que le chlore joue le rôle de catalyseur de l'effet photoélectrique. Le mécanisme est illustré sur la \autoref{fig:catalyseur}.
Le champs de rayonnement UV, intense en bord de nuage atomique, photoionise le chlore et produit des ions $\mathrm{Cl}^+$ et des électrons. Le transfert de charge du $\mathrm{Cl}^+$ avec l'hydrogène est une réaction rapide qui permet au chlore de se rendre de nouveau disponible pour la photoionisation. Le chlore permet ainsi de ioniser indirectement l'hydrogène. Par conséquent, la fraction électronique du gaz augmente.  
Or on sait que l'effet photoélectrique sur les grains fonctionne d'autant plus que la fraction électronique dans le nuage est importante \footnote{Une forte densité d'électrons rend plus facile la recombinaison électronique des grains ce qui maintient le degré d'ionisation des grains raisonnablement faible. Il est plus facile d'arracher un électron d'un grain neutre que d'un grain qui a déjà été ionisé.}. 
L'effet photoélectrique chauffe ainsi le gaz ce qui améliore l'efficacité du transfert de charge du $\mathrm{Cl}^+$ avec l'hydrogène. 
En d'autres termes, le chlore induit une rétroaction positive de l'effet photoélectrique sur les grains. Cet emballement a tendance à chauffer le gaz à des températures nettement plus fortes et ce malgré la faible abondance du chlore. \newline

\begin{figure}[b!]
   \centering
        \includegraphics[trim = {2cm 5cm 4cm 4cm},clip, width=0.8\textwidth]{figure/Cl/Cl_heating_fr-5.pdf}
    \caption{Schéma représentant l'impact du chlore sur la chimie de bord de nuage atomique}
    \label{fig:catalyseur}
\end{figure}{}

L'emballement de l'effet photoélectrique se produit à partir de 1000 K. Or la photoionisation du carbone et du soufre produit les électrons en bord de nuage atomique indépendamment de la température. Il existe une température où le transfert de charge devient suffisamment efficace pour que la fraction d'électrons créée via le chlore devienne dominante devant celle de la ionisation du carbone et du soufre et amorce la rétroaction de l'effet photoélectrique. L'amplification dépend donc l'énergie d'activation du transfert de charge $\mathrm{Cl}^+  + \mathrm{H}    \rightarrow \mathrm{Cl}   +  \mathrm{H}^+$ qui vaut 6290 K. \newline

Parmi les espèces figurant dans le modèle, la carbone, le soufre, le silicium ou le fer ont également un potentiel de ionisation inférieur à celui de l'hydrogène (\autoref{tab:gaz}). Pourtant aucun ne peut effectuer un transfert de charge avec l'hydrogène qui est l'espèce majoritaire en bord de nuage atomique. Ces espèces ne peuvent pas donc pas provoquer un emballement similaire à celui induit par le chlore. 



%%%%%%%%%%%%%%%%%%%%%%%%%%%%%%%%%%%%%%%%%%%%%%%%%%%%%%%%%%%%%%%%%%%%%%%%%%%%%%%%%%%%%%%%%%%%%%%

\subsection{Modèle analytique - Chimie}

Les espèces qui contribuent à la production d'électrons en bord de nuage sont les ions hydrogènes $\mathrm{H}^+$, carbones $\mathrm{C}^+$ et soufres $\mathrm{S}^+$. On peut supposer qu'en entrée de nuage atomique le carbone et le soufre sont ionisés ce qui fournit une fraction d'électrons minimale de $10^{-4}$. Le modèle doit retrouver l'augmentation de la densité d'électrons (jusqu'à $2\ 10^{-3}$) pour des températures supérieures à 1000 K.
 
\subsubsection{Ion hydrogène}

J'ai isolé les réactions principales qui font intervenir les ions $\mathrm{H}^+$. Les réactions avec l'oxygène sont négligées car la formation et destruction de $\mathrm{H}^+$ par l'oxygène se compensent totalement en première approximation. 

\begin{equation}
    \begin{array}{lccccclr}
        \mathrm{Cl}^+ & + &\mathrm{H}   & \rightarrow &\mathrm{Cl}  & + & \mathrm{H}^+ & (k_3) \\
        \mathrm{Cl}  & + & \mathrm{H}^+  & \rightarrow & \mathrm{Cl}^+ & + &\mathrm{H}  & (k_4) \\
        \mathrm{H}^+  & + & \mathrm{e}^-  & \rightarrow &\mathrm{H}   &   &     & (k_5) \\
    \end{array}
\end{equation}

A l'état stationnaire, le bilan de formation des ions $\mathrm{H}^+$ donne
\begin{equation}\label{eq:h+}
    \frac{d}{dt}n(\mathrm{H}^+) = k_3n(\mathrm{Cl}^+)n(\mathrm{H}) - k_4n(\mathrm{Cl})n(\mathrm{H}^+) - k_5 n(\mathrm{H}^+)n(\mathrm{e}^-) = 0
\end{equation}

En introduisant la fraction atomique de chlore $\delta_{Cl}$, fixée dans le gaz, et le bilan de charge on obtient une équation en $n(\mathrm{H}^+)$ :

\begin{equation}
    -k_3n(\mathrm{Cl}^+)n_{\mathrm{H}} + \bigg( \frac{k_3 k_4}{k_1} n_{\mathrm{H}} n(\mathrm{Cl}^+) + k_5 \big(n(\mathrm{C}^+)+ n(\mathrm{S}^+)\big) \bigg) n(\mathrm{H}^+) + k_5 n(\mathrm{H}^+)^2 = 0
\end{equation}

 avec,
\begin{equation}
    \delta_{Cl} = \SI{1.8}{10^{-7}} = \frac{n(\mathrm{Cl}) + n(\mathrm{Cl}^+) + ...}{n(\mathrm{H}) + n(\mathrm{H}^+) + 2n(\mathrm{H}_2) + ...} \approx \frac{1}{n_{\mathrm{H}}} (n(\mathrm{Cl}) + n(\mathrm{Cl}^+) )
\end{equation}

On obtient une solution qui dépend de $n(\mathrm{Cl}^+)$,
\begin{equation}
\resizebox{1.1\hsize}{!}{
    \boxed{n(\mathrm{H}^+) = -\frac{1}{2} \bigg( \frac{k_3 k_4}{k_1 k_5} n_{\mathrm{H}} n(\mathrm{Cl}^+) + n(\mathrm{C}^+)+ n(\mathrm{S}^+) \bigg) \pm \frac{1}{2} \sqrt{\bigg( \frac{k_3 k_4}{k_1 k_5} n_{\mathrm{H}} n(\mathrm{Cl}^+) + n(\mathrm{C}^+)+ n(\mathrm{S}^+) \bigg)^2 + 4\frac{k_3}{k_5}n_{\mathrm{H}} n(\mathrm{Cl}^+)}}
    }
\end{equation}

% \subsubsection{Rôle de l'oxygène}
% Au vu des taux de réactions impliquant le $\mathrm{H}^+$ nous pourrions choisir d'inclure la chimie de $\mathrm{H}^+$ avec l'oxygène. A hautes températures, les taux de $ \mathrm{H}^+ + O \leftrightarrows\mathrm{H}+ O^+ \quad (k_6,k_7)$ prédominent sur les autres réactions. Si l'on considère que ces réactions pour la chimie de l'oxygène, les $\mathrm{H}^+$ produits seront consommés pour former du $H$. Rien ne se passe. Pour s'en convaincre il suffit d'écrire la nouvelle équation bilan de $\mathrm{H}^+$ et d'utiliser celle pour $O^+$.

% \begin{equation}
%     \frac{d}{dt}n(O^+) = k_6 n(\mathrm{H}^+)n(O) - k_7n(\mathrm{H})n(O^+) = 0
% \end{equation}

% En l'injectant dans le nouveau bilan de formation pour le $\mathrm{H}^+$, ces nouveaux termes disparaissent. Cependant si l'on calcule l'expression de $n(\mathrm{H}^+)$ à partir des concentrations de $n(O^+)$ et $n(O)$, l'approximation donne une meilleure correspondance avec l'abondance calculé par le code. 

% \begin{equation}
%     n(\mathrm{H}^+) =  \frac{k_3 n_{\mathrm{H}} n(\mathrm{Cl}^+) + k_6 n_{\mathrm{H}} n(O^+) }{k_5 n(\mathrm{e}^-) + k_4 \frac{n(\mathrm{Cl}^+)}{A} + k_7 n(O) }
% \end{equation}

% Utiliser les abondances de l'oxygène demande d'étudier sa chimie, et donc de prendre en compte d'autres espèces comme $OH$ ou $O_2$ ce qui complexifie le modèle. 

%%%%%%%%%%%%%%%%%%%%%%%%%%%%%%%%%%%%%%%%%%%%%%%%%%%%%%%%%%%%%%%%%%%%%%%%%%%%%%%%%%%%%%%%%%%%%%

\subsubsection{Ion chlore}

La recombinaison électronique de $\mathrm{Cl}^+$ reste négligeable devant le transfert de charge avec $\mathrm{H}$ pour des températures supérieure à $100$K et la destruction par formation du $\mathrm{HCl}^+$ devient dominante dans la gamme de température $100-1000$K. Négliger ces réactions donne une approximation correcte de la densité d'ion chlore pour les régimes de températures $T\geq 1000$ K et garde l'emballement de l'effet photoélectrique. On considère ainsi les réactions ci-dessus :


\begin{equation}
    \begin{array}{lccccclr}
       \mathrm{Cl}  & + & h\nu & \rightarrow & \mathrm{Cl}^+ & + & \mathrm{e}^- & (k_1) \\
        \mathrm{Cl}^+ & + &\mathrm{H}   & \rightarrow &\mathrm{Cl}  & + & \mathrm{H}^+ & (k_3) \\
    \end{array}
\end{equation}

Un travail similaire nous donne 
\begin{equation}
    k_1(\xi_{Cl}n_{\mathrm{H}} - n(\mathrm{Cl}^+)) - k_3 n(\mathrm{Cl}^+) n_{\mathrm{H}} = 0
\end{equation}

Ce qui fait en posant $A = \frac{k_1}{k_3 n_{\mathrm{H}}}$:

\begin{equation}
\boxed{n(\mathrm{Cl}^+) = \frac{k_1 \xi_{Cl} n_{\mathrm{H}}}{k_1 + k_3 n_{\mathrm{H}}} = \frac{A}{1 + A} \xi_{cl} n_{\mathrm{H}}}
\end{equation}


%%%%%%%%%%%%%%%%%%%%%%%%%%%%%%%%%%%%%%%%%%%%%%%%%%%%%%%%%%%%%%%%%%%%%%%%%%%%%%%%%%%%%%%%%%%%%%%

\subsection{Recombinaison de $\mathrm{H}^+$ sur les grains}

La recombinaison du $\mathrm{H}^+$ sur les grains est calculé dans le code à sa manière (type 14). Le modèle chimique sans la recombinaison surestime la densité de protons aux hautes températures. \cite{Weingartner_2001} donne un taux de recombinaison $\alpha$ (équations [5][8]). 

\begin{equation}
    \alpha_{g}\left(\mathrm{X}^{i}, \psi, T\right) \approx \frac{10^{-14} C_{0} }{1+C_{1} \psi^{C_{2}}\left(1+C_{3} T^{C_{4}} \psi^{-C_{5}-C_{6} \ln T}\right)} \quad \mathrm{cm}^{3} \mathrm{s}^{-1}
\end{equation}

avec $\psi = \frac{G_0 \sqrt{T}}{n_e}$. Les coefficients $\mathrm{C}_i$ sont des paramètres donnés dans l'article. Le bilan de formation devient : 

\begin{equation}\label{eq:h+}
    \frac{d}{dt}n(\mathrm{H}^+) = k_3n(\mathrm{Cl}^+)n(\mathrm{H}) - k_4n(\mathrm{Cl})n(\mathrm{H}^+) - k_5 n(\mathrm{H}^+)n(\mathrm{e}^-) -\alpha_{g} n(\mathrm{H}^+)n_{\mathrm{H}} = 0
\end{equation}

On obtient une équation implicite en fonction de la densité d'électrons. La solution $n_e$ vérifie :
\begin{equation}
    x = n(\mathrm{H}^+)(x) + n(\mathrm{S}^+)  + n(\mathrm{C}^+)
\end{equation}

La fonction \texttt{newton} de la librairie \texttt{scipy.optimize} vient à bout de ce système. La densité d'électrons calculée montré sur la figure \ref{fig:Cl:model:rec}. On a résolu le système dans deux cas. Un cas avec les PAH (fraction massique = $4.6\,10^{-2}$) où le taux de recombinaison donné \cite{Weingartner_2001} donne une bonne estimation de la densité d'électrons. Un second cas sans PAH, où il faut réduire d'un facteur $1/6$ la recombinaison de $\mathrm{H}^+$ sur les grains. On avait toujours $r_{\mathrm{min}} = 1\,10^{-7}\,\mathrm{m}$ et $r_{\mathrm{max}} = 3\,10^{-5}\,\mathrm{m}$.

\textit{Quel réaction réduit la densité d'électrons à très hautes températures et qui nous empêche d'avoir la tangente horizontale ?}


\begin{figure}[htbp]
    \centering
    \begin{subfigure}[t]{0.45\textwidth} % "0.45" donne ici la largeur de l'image
        \centering \includegraphics[trim = {0 0 0 1.5cm},clip,width=1\textwidth]{figure/Cl/model/test_calc_PAH_e.png}
        \caption{avec PAH}
    \end{subfigure}
    ~ 
    \begin{subfigure}[t]{0.45\textwidth}
        \centering \includegraphics[trim = {0 0 0 1.5cm},clip,width=1\textwidth]{figure/Cl/model/test_calc_e.png}
        \caption{sans PAH}
    \end{subfigure}
    \caption{Profil de la densité électronique calculé par le modèle. La formule proposée par \cite{Weingartner_2001} prend en compte la recombinaison sur les PAH. Avec la recombinaison, la prédiction de la densité d'électrons est bonne dans le cas avec PAH. Si on ne prend pas en compte les PAH, il faut diminuer le taux de recombinaison d'un facteur $1/6$ pour prédire une densité d'électrons raisonnable.}
    \label{fig:Cl:model:rec}
\end{figure}



%%%%%%%%%%%%%%%%%%%%%%%%%%%%%%%%%%%%%%%%%%%%%%%%%%%%%%%%%%%%%%%%%%%%%%%%%%%%%%%%%%%%%%%%%%%%%%%
\subsection{Modèle analytique - Thermique}

On cherche la température d'équilibre du milieu dans différentes configurations de PDR. On considère dans ce modèle uniquement le chauffage par effet photoélectrique, le chauffage par $\mathrm{H}_2$ qui joue principalement dans les régions hautes densités et faible champs de rayonnement sera traité plus tard. On cherche $\Gamma = \Lambda $ en fonction de la température avec

\begin{equation}
    \begin{split}
        \Gamma &= \Gamma_{pe}^{\mathrm{Rollïg}} \\
        \Lambda &=   \Lambda_{\mathrm{CII}\ 158 \mu \mathrm{m}} + \Lambda_{\mathrm{OI}\ 63 \mu \mathrm{m}} + \Lambda_{\mathrm{OI}\ 146 \mu \mathrm{m}}  + \Lambda_{\mathrm{H}_\alpha} + \Lambda_{\mathrm{g}-\mathrm{g}} + \Lambda_{\mathrm{rec}}^{\mathrm{Rollïg}}
    \end{split}
\end{equation}


\subsubsection{Effet photoélectrique sur les grains}

De \cite{Rollig2005} (Eq (10) ou (C.3)) qui provient de \cite{BakesTielens1994}, sans PAH, on utilise : 

\begin{equation}
    \Gamma_{\mathrm{pe}} = 10^{-24}\,G_0\,n_\mathrm{H}\, \frac{2\times 10^{-2}}{1 + 2\times 10^{-4}\,G_0 \sqrt{T}/n_e} \operatorname{erg} \mathrm{cm}^{-3} \mathrm{s}^{-1}
\end{equation}

Avec $G_0 = 1.71\chi \times 0.5$ car il considère une illumination provenant que d'un coté. (\cite{Wolfire_2003} Eq 20, \cite{BakesTielens1994}) propose une autre formule prenant en compte les PAH qui est de la forme 

\begin{equation}
    \Gamma_{\mathrm{pe}}^{\mathrm{Wolf}} = 10^{-24}\,G_0\,n_\mathrm{H}\, \bbig[ \frac{4.9\times 10^{-2}}{1 + 4\times 10^{-3}\, \frac{G_0 \sqrt{T}}{n_e \phi_\mathrm{PAH}}} + \frac{3.7\times 10^{-2} (T/10^4)^{0.7}}{1 + 2\times 10^{-4}\, \frac{G_0 \sqrt{T}}{n_e \phi_\mathrm{PAH}}} \bbig] \operatorname{erg} \mathrm{cm}^{-3} \mathrm{s}^{-1}
\end{equation}

avec $\phi_\mathrm{PAH}$ une efficacité de collision compris entre 0 et 1. L'effet photoélectrique est d'autant plus efficace sur les petits grains \cite{DraineBook}. Le choix de $r_\mathrm{min}$ (la taille minimale des grains dans la description MRN) est décisif. Si l'on trace ces formules sur la \autoref{fig:Cl:pePAH} pour différent $\phi_\mathrm{PAH}$ et $r_\mathrm{min}$ on voit que la formule de \cite{Rollig2005} sous-estime le taux calculé par le code PDR de Meudon. Avec $r_\mathrm{min} = 1\,10^7\,\mathrm{nm}$, la prescription de Rollig est 3 fois plus petit que le chauffage calculé par Meudon. Si l'on augmente la taille de grain minimale, on voit que cette écart diminue ce qui est normale car l'on réduit l'efficacité de l'effet photo-électrique en enlevant les petits grains. La prescription de \cite{Wolfire_2003} laisse plus de liberté à travers le $\phi_\mathrm{PAH}$ (not much to say).

\begin{figure}[!h]
    \centering
    \begin{subfigure}[t]{0.45\textwidth} % "0.45" donne ici la largeur de l'image
        \centering \includegraphics[trim = {0 0 0 1cm},clip,width=1\textwidth]{figure/Cl/pePAH/pe_formulae_rgmin3.png}
        \caption{$r_\mathrm{min} = 3\,10^{-7} \, \mathrm{nm}$}
    \end{subfigure}
    ~ 
   \begin{subfigure}[t]{0.45\textwidth} % "0.45" donne ici la largeur de l'image
        \centering \includegraphics[trim = {0 0 0 1cm},clip,width=1\textwidth]{figure/Cl/pePAH/pe_formulae_rgmin5.png}
        \caption{$r_\mathrm{min} = 5\,10^{-7}\, \mathrm{nm}$}
    \end{subfigure}
    \caption{Comparaison des formules de l'effet photoélectrique de \cite{Rollig2005} et \cite{Wolfire_2003} avec le code PDR utilisant différents $r_\mathrm{min}$}
    \label{fig:Cl:pePAH}
\end{figure}

%%%%%%%%%%%%%%%%%%%%%%%%%%%%%%%%%%%%%%%%%%%%%%%%%%%%%%%%%%%%%%%%%%%%%%%%%%%%%%%%%%%%%%%%%%%%%%%

\subsubsection{Couplage gaz-grains}

De \cite{Rollig2005} ($Z=1$), le couplage gaz grain s'exprime,

\begin{equation}
    \Lambda_{\mathrm{g}-\mathrm{g}} = 3.5\,10^{-34}\times \sqrt{T}(T - T_g) {n_\mathrm{H}}^2 \operatorname{erg} \mathrm{cm}^{-3} \mathrm{s}^{-1}
\end{equation}

Où $T_g$ est donné par Eq. 6 de \cite{HollenbachTakahashiTielens_1991}
\begin{equation}
    T_g = 12.2 \,{G_0}^{0.2}
\end{equation}

%%%%%%%%%%%%%%%%%%%%%%%%%%%%%%%%%%%%%%%%%%%%%%%%%%%%%%%%%%%%%%%%%%%%%%%%%%%%%%%%%%%%%%%%%%%%%%%


\subsubsection{Refroidissement par les raies d'émissions du $\mathrm{C}^+$ et de $\mathrm{O}$}

[CII]$158 \mu \mathrm{m}$, provient de \cite{Rollig2005}, Equation (A.2) ($Z=1$)

\begin{equation}
    \Lambda_{\mathrm{CII}\ 158   \mu \mathrm{m}}= n(\mathrm{C}^+) \frac{2.89 \times 10^{-20}}{1+\frac{1}{2} \exp (92 / T)\left(1+\frac{1300}{n_\mathrm{H}}\right)} \operatorname{erg} \mathrm{cm}^{-3} \mathrm{s}^{-1}
\end{equation}

Pour les raies de [OI], Rollïg prend en compte les transitions adjacentes de [OI]$62 \mu \mathrm{m}$ et $[OI]146 \mu \mathrm{m}$. ($Z$,$\beta = 1$)


\begin{equation}
\begin{split}
    \Lambda_{\mathrm{OI}\ 63 \mu \mathrm{m}} &= 3.15\,10^{-14} \times 8.46\,10^{-5} \times 
    n(\mathrm{O}) \\
    & \times \frac{e^{98/T} 3 n_\mathrm{H} (n_\mathrm{H} + \beta\, n_{\mathrm{cr}_{01}} ) }{{n_\mathrm{H}}^2+ e^{98/T}(n_\mathrm{H} + \frac{1}{2} n_{\mathrm{cr}_{01}} ) (3 n_\mathrm{H} + 5\, e^{228/T} (n_\mathrm{H} + \frac{1}{2} n_{\mathrm{cr}_{12}} )) } \operatorname{erg} \mathrm{cm}^{-3} \mathrm{s}^{-1}
\end{split}
\end{equation}

\begin{equation}
\begin{split}
    \Lambda_{\mathrm{OI}\ 146 \mu \mathrm{m}} &= 1.35\,10^{-14} \times 8.46\,10^{-5} \times 
    n(\mathrm{O}) \\
    & \times \frac{ {n_\mathrm{H}}^2 }{n_\mathrm{H}^2+ e^{98/T}(n_\mathrm{H} + \frac{1}{2} n_{\mathrm{cr}_{01}} ) (3 n_\mathrm{H} + 5\, e^{228/T} (n_\mathrm{H} + \frac{1}{2} n_{\mathrm{cr}_{12}} )) } \operatorname{erg} \mathrm{cm}^{-3} \mathrm{s}^{-1}
\end{split}
\end{equation}

Avec $n_{\mathrm{cr}_{01}}(T) = \frac{1.66\,10^{-5} }{1.35\,10^{-11} T^{0.45}} $ et $n_{\mathrm{cr}_{12}}(T) = \frac{8.46\,10^{-5} }{4.37\,10^{-12} T^{0.66}} $

%%%%%%%%%%%%%%%%%%%%%%%%%%%%%%%%%%%%%%%%%%%%%%%%%%%%%%%%%%%%%%%%%%%%%%%%%%%%%%%%%%%%%%%%%%%%%%%

\subsubsection{Emission Lyman $\mathrm{H}\alpha$}

\cite{tielens2005}, Eq 2.62

\begin{equation}
    \Lambda_{\mathrm{H}\alpha} = 7.3\, 10^{-19}\,n_e\,n_\mathrm{H}\,e^{-118400/T} \operatorname{erg} \mathrm{cm}^{-3} \mathrm{s}^{-1}
\end{equation}

%%%%%%%%%%%%%%%%%%%%%%%%%%%%%%%%%%%%%%%%%%%%%%%%%%%%%%%%%%%%%%%%%%%%%%%%%%%%%%%%%%%%%%%%%%%%%%%

\subsubsection{Recombinaison des électrons sur les grains}

Dans le modèle de chlore, on prend celle de \cite{Rollig2005} Eq.4  qui provient elle même de \cite{BakesTielens1994}.

\begin{equation}
    \Lambda_{\mathrm{rec}}^{\mathrm{Rollïg}} = 3.49\,10^{-30} T^{0.944} \, (\frac{G_0 \sqrt{T}}{n_e})^{\frac{0.735}{T^{0.068}}} \, n_e \, n_\mathrm{H} \operatorname{erg} \mathrm{cm}^{-3} \mathrm{s}^{-1}
\end{equation}

Il existe également celle ci \cite{WolfireHollenbachMcKeeTielensBakes_1995} Eq. 9 que l'on n'utilise pas.

\begin{equation}
    \Lambda_{\mathrm{rec}}^{\mathrm{Wolfire}} = 4.65\,10^{-30} T^{0.94} \, (\frac{G_0 \sqrt{T}}{n_e})^{\frac{0.74}{T^{0.068}}} \, n_e \, n_\mathrm{H} \operatorname{erg} \mathrm{cm}^{-3} \mathrm{s}^{-1}
\end{equation}

%%%%%%%%%%%%%%%%%%%%%%%%%%%%%%%%%%%%%%%%%%%%%%%%%%%%%%%%%%%%%%%%%%%%%%%%%%%%%%%%%%%%%%%%%%%%%%%

\subsection{Prédiction de la température au bord de nuage}
\subsubsection{Carte de température en bord atomique du nuage}

Que fait le modèle ? On trace en fonction de la température le terme de chauffage $\Gamma$ et $\Lambda$ à partir de la densité d'électrons $n_e$ que l'on obtient à partir du modèle chimique. On cherche les $T_{eq}$ tel que $\Gamma - \Lambda  = 0$ que l'on caractérise de stable si autour du point d'équilibre $\frac{d}{dT}(\Gamma - \Lambda) <0$ et d'instable sinon. On visualise sur \autoref{fig:Cl:grid:mapT} la température maximale atteignable par le bord atomique. La zone hachurée indique les solutions instables.

\textit{Conséquences dynamiques : Hydra d'Emeric}

\begin{figure}[htbp]
    \centering
    \begin{subfigure}[t]{0.45\textwidth} % "0.45" donne ici la largeur de l'image
        \centering \includegraphics[trim = {0 0 0 1.5cm},clip,width=1\textwidth]{figure/Cl/model/mapG0nHTeq_pe_OI_CII_ggr_elecrec_lyman_OI_imp.png}
        \caption{avec chlore}
    \end{subfigure}
    ~ 
    \begin{subfigure}[t]{0.45\textwidth}
        \centering \includegraphics[trim = {0 0 0 1.5cm},clip,width=1\textwidth]{figure/Cl/model/mapG0nHTeq_pe_OI_CII_ggr_elecrec_lyman_OI_noCl.png}
        \caption{sans chlore}
    \end{subfigure}
    \caption{Carte de la température en bord de nuage atomique ($A_\mathrm{V}= 10^{-6}$) prédit par le modèle.}
    \label{fig:Cl:model:mapT}
\end{figure}

Si l'on prend en compte la recombinaison sur les grains, la prédiction des la densité d'électrons devient quantitavement proche de celle que calculerait le modèle. En retracant les cartes de températures (\autoref{}) on se rend compte qu'elles ne sont pas beaucoup modifiées.

\subsubsection{Grille de modèles}

Les grilles de modèle de la version du code \uncinq donne des cartes de température plotée sur la \autoref{fig:Cl:grid:mapT}. Le même quadrant est impacté bien que la solution haute ne soit pas toujours atteinte. La région faible champs de rayonnement et haute densité est prédominée par le chauffage par la molécule $\mathrm{H}_2$ qui n'est pas pris en compte dans ce petit modèle.  

\textit{Que se passe t'il à faible densité ? Pourquoi n'a t on pas les tangentes horizontales de T ?}

\begin{figure}[!htbp]
    \centering
    \begin{subfigure}[t]{0.45\textwidth} % "0.45" donne ici la largeur de l'image
        \centering \includegraphics[trim = {0 0 0 1.5cm},clip,width=1\textwidth]{figure/Cl/grid/mapTCl.png}
        \caption{avec chlore}
    \end{subfigure}
    ~ 
    \begin{subfigure}[t]{0.45\textwidth}
        \centering \includegraphics[trim = {0 0 0 1.5cm},clip,width=1\textwidth]{figure/Cl/grid/mapT.png}
        \caption{sans chlore}
    \end{subfigure}
    \caption{Température en bord de nuage atomique ($A_\mathrm{V}= 10^{-6}$) calculé par le code PDR \uncinq. Les PAH et la recombinaison du $\mathrm{H}^+ $ sur les grains n'ont pas été pris en compte.}
    \label{fig:Cl:grid:mapT}
\end{figure}

On a discuté précédemment de la prescription de Rollig qui sous-estime l'efficacité de l'effet photoélectrique pour des modèles de PDR avec une taille minimale de grains de $1\,10^{-7} \, \mathrm{nm}$. Si l'on corrige d'un facteur 3 le taux de chauffage c.a.d. que l'on prend un $\Gamma_{pe}^{'} = 3 \times \Gamma_{pe}^{Rollïg}$, on obtient une nouvelle carte de température assez différente.


\subsubsection{Chauffage à l'entrée du nuage moléculaire}

Il arrive qu'en entrée du nuage moléculaire, la température du gaz augmente de manière irrégulière. Dans cette région, l'effet photoélectrique devient encore plus efficace en raison de l'augmentation de la fraction électronique du nuage. La recombinaison sur les grains est rapide s'il y a une quantité importante d'électron autour des grains. Dans cette zone, le $\mathrm{H}_2$ tente de se former. La voie principale de formation passe par l'association radiative :

\begin{equation}
    \begin{array}{lccccclr}
       \mathrm{H}  & + & e^- & \rightarrow & \mathrm{H}^{-} & + & h\nu &  \\
    \end{array}
\end{equation}

suivi d'un détachement associatif :

\begin{equation}
    \begin{array}{lcccccclr}
       \mathrm{H}  & + &  \mathrm{H}^- & \rightarrow & \mathrm{H}_2 & + & e^- & + & KE\\
    \end{array}
\end{equation}

Cette réaction peut produire des électrons en quantité si elle devient dominante ce qui est le cas autour de $A_\mathrm{v} \approx 0.5 \mathrm{mag}$ sur \autoref{fig:Cl:grid:proH}. La formation du $\mathrm{H}_2$ déclenche également une chimie chaude permettant notamment de former le doux $\mathrm{CH}^+$ que l'on observe. Cette augmentation de température ne doit pas être confondue avec l'emballement provoqué par le Chlore.


\begin{figure}[!htbp]
    \centering
        \centering \includegraphics[trim = {0 0 0 1.5cm},clip,width=0.6\textwidth]{figure/Cl/grid/profil_H.png}
        \caption{...}
    \label{fig:Cl:grid:proH}
\end{figure}

\subsection{Diagramme d'état}

\subsection{Traceurs modifiés par le chlore}

On a pu extraire à partir des grilles calculé par le code (\uncinq) les raies d'émissions de chaques modèles et tracer des cartes d'intensités dans l'espace des paramètres et visualiser les spectres des traceurs impactés par l'ajout du chlore dans la chimie. On a étudié de 5 calculs de PDR, avec ou sans chlore, ayant des conditions physiques différentes afin de comparer localement le modèle au code de Meudon. 

\begin{figure}[!htbp]
    \centering
        \centering \includegraphics[trim = {0 0 0 1.5cm},clip,width=0.6\textwidth]{figure/Cl/grid/mapT_cross.png}
        \caption{...}
\end{figure}

On a compris que l'ajout du chlore : pas de signature, modifiaction des raies, facteur 3..

\subsubsection{Traceurs impactés}

Les traceurs impactés par l'ajout du chlore sont $\mathrm{N}$, $\mathrm{N}^+$, $\mathrm{S}$ et $\mathrm{Si}$. \autoref{fig:Cl:gridModelEmiss:yes}


\begin{figure}[!htbp]
    \centering
    \begin{subfigure}[t]{0.45\textwidth} % "0.45" donne ici la largeur de l'image
        \centering \includegraphics[trim = {0 0 0 1.5cm},clip,width=1\textwidth]{figure/Cl/gridModelEmiss/I_comp_Np.png}
        \caption{Spectre de $\mathrm{N}^+$}
    \end{subfigure}
    ~ 
   \begin{subfigure}[t]{0.45\textwidth} % "0.45" donne ici la largeur de l'image
        \centering \includegraphics[trim = {0 0 0 1.5cm},clip,width=1\textwidth]{figure/Cl/gridModelEmiss/I_comp_N.png}
        \caption{Spectre de $\mathrm{N}$}
    \end{subfigure}
    
    \begin{subfigure}[t]{0.45\textwidth} % "0.45" donne ici la largeur de l'image
        \centering \includegraphics[trim = {0 0 0 1.5cm},clip,width=1\textwidth]{figure/Cl/gridModelEmiss/I_comp_S.png}
        \caption{Spectre de $\mathrm{S}$}
    \end{subfigure}
    ~
    \begin{subfigure}[t]{0.45\textwidth} % "0.45" donne ici la largeur de l'image
        \centering \includegraphics[trim = {0 0 0 1.5cm},clip,width=1\textwidth]{figure/Cl/gridModelEmiss/I_comp_Si.png}
        \caption{Spectre de $\mathrm{Si}$}
    \end{subfigure}
    
    \caption{Spectre chlore code PDR \uncinq}
    \label{fig:Cl:gridModelEmiss:yes}
\end{figure}



\subsubsection{$\mathrm{CS}$, $\mathrm{H}_2\mathrm{O}$}


\begin{figure}[!htbp]
    \centering
    \begin{subfigure}[t]{0.45\textwidth} % "0.45" donne ici la largeur de l'image
        \centering \includegraphics[trim = {0 0 0 1.5cm},clip,width=1\textwidth]{figure/Cl/gridModelEmiss/I_comp_CS.png}
        \caption{Spectre de $\mathrm{CS}$}
    \end{subfigure}
    ~ 
   \begin{subfigure}[t]{0.45\textwidth} % "0.45" donne ici la largeur de l'image
        \centering \includegraphics[trim = {0 0 0 1.5cm},clip,width=1\textwidth]{figure/Cl/gridModelEmiss/I_comp_H2O.png}
        \caption{Spectre de $\mathrm{H}_2\mathrm{O}$}
    \end{subfigure}
    
    \begin{subfigure}[t]{0.45\textwidth} % "0.45" donne ici la largeur de l'image
        \centering \includegraphics[trim = {0 0 0 1.5cm},clip,width=1\textwidth]{figure/Cl/gridModelEmiss/I_comp_H2.png}
        \caption{Spectre de $\mathrm{H}_2$}
    \end{subfigure}
 
    
    \caption{Spectre chlore code PDR \uncinq}
    \label{fig:Cl:gridModelEmiss:no}
\end{figure}


\subsection{Comparaison de quelques modèles}


%%%%%%%%%%%%%%%%%%%%%%%%%%%%%%%%%%%%%%%%%%%%%%%%%%%%%%%%%%%%%%%%%%%%%%%%%%%%%%%%%%%%%%%%%%%%%%%

\clearpage
\section{Pompage UV de la molécule $\mathrm{H}_2$}


On cherche dans un premiers temps des expressions permettant d'estimer le chauffage par pompage UV de la molécule $\mathrm{H}_2$ calculé par le code. 

\subsection{Chauffage net par Rollïg (1995)}

\subsubsection{Chauffage par désexcitation collisionelle}
Eq (10) ou (C.3).

Il propose un modèle d'excitation effectif à deux niveaux au lieu d'un modèle prenant en compte les 15 premiers états vibrationels ($v\leq 15$) de la molécule. Se fonde sur Sternberg&Dalgarno (1995) et Burton (1990). Traduit l'absorption du flux de photons à un taux $P$ (90\% sert à à exciter la molécule) puis le chauffage par désexcitation collisionelle. 
$\Gamma_{\mathrm{H}_2^\star} > 0$
 
\begin{equation}
\Gamma_{\mathrm{H}_2^\star} = n_{\mathrm{H}_2}\frac{\chi P}{1 + \frac{A_{\text{eff}}+ \chi D_{\text{eff}}}{\gamma_{\text{eff}}\, n}} \Delta E \operatorname{erg} \mathrm{cm}^{-3} \mathrm{s}^{-1}
\end{equation}

Où le taux de pompage par unité de champs FUV $P = 2 \times 2.9\,10^{-10}\,\mathrm{s}^{-1}$. Le facteur 2 est considéré car Rollig considère un demi champs incident en bord de nuage (contrairement à nous qui utilisons le flux calculé à partir du transfert radiatif total). Le taux de pompage représente $100\%$ du pompage. Normalement $85$-$90\%$ du pompage maintient la molécule dans un état excité, les autres desexcitation dissocient la molécule. Il retranche un terme de refroidissement pour corriger.

\subsubsection{Refroidissement par émission de photons}

Eq (11) ou (C.2).
Excitation collisonelle (à $v=1$) puis émission (ou photodissociation à inspecter : pourquoi il prend en compte l). Rollïg veut un terme de refroidissement et construit le taux tel que $\Lambda_{\mathrm{H}_2} < 0$.

\begin{equation}
    \Lambda_{\mathrm{H}_2} = - \Delta E_{1,0} \, \gamma_{1,0} e^{-\Delta E_{1,0}/kT} n_{\mathrm{H}} \, n_{\mathrm{H}_2} \frac{A_{1,0} + \chi D_1}{\gamma_{1,0}  n_{\mathrm{H}} + A_{1,0} + \chi D_1} \operatorname{erg} \mathrm{cm}^{-3} \mathrm{s}^{-1}
\end{equation}

\subsection{Bialy et Sternberg (2019)}
\subsubsection{Chauffage par désexcitation collisionelle}

\cite{BialySternberg_2019} (A12) considère que 9 photons sur 10 permettent d'avoir du $\mathrm{H}_2$ excité. 

\begin{equation}
    \Gamma_{\mathrm{H}_2 \, \mathrm{pump}} = 9D_0 \chi \, E_{\mathrm{pump}} \, n(\mathrm{H}_2) \times \frac{1}{1 + \frac{n_{\mathrm{crit}}}{n_\mathrm{H}} } \operatorname{erg} \mathrm{cm}^{-3} \mathrm{s}^{-1}
\end{equation}

$D_0$ est le taux de photodissociation ($P \approx 10 D_0$, 1 photon sur 10 mène à une photodissociation). $E_{\mathrm{pump}} = 1.12\,eV$ et $n_{\mathrm{crit}} = 1.1\,10^5 /\sqrt{T/1000\mathrm{K}} \, \mathrm{cm}^{-3}$ représente la compétition entre émission spontanée et désexcitation par collisions. Si on compare ces termes entre \cite{BialySternberg_2019} et \cite{Rollig2005} au bord de nuage avec un champs de rayonnement $\chi = 10^5 $

\begin{equation}
    \frac{A_{\text{eff}}+ \chi D_{\text{eff}}}{\gamma_{\text{eff}}} \approx 2.9\,10^{6} /\sqrt{T/1000\mathrm{K}} \, \mathrm{cm}^{-3} \quad !\sim n_{\mathrm{crit}}
\end{equation}

\subsubsection{Chauffage par photodissociation}

\begin{equation}
    \Gamma_{\mathrm{H}_2 \, \mathrm{pd}} = D_0\,\chi \, E_\mathrm{pd} \, n(\mathrm{H}_2)
\end{equation}

\subsection{Absorption du rayonnement}
\subsubsection{Shielding par le molécule $\mathrm{H}_2$}

On utilise la fonction fitté (Eq.37) par \cite{DraineBertoldi_1996} pour calculer le \textit{shielding} de la molécule $\mathrm{H}_2$ sur le taux d'excitation par pompage $\chi P$.

$$
f_{\text {shield}}\left(x\right)=\frac{0.965}{\left(1+x / b_{5}\right)^{2}}+\frac{0.035}{(1+x)^{0.5}} \exp \left[-8.5 \times 10^{-4}(1+x)^{0.5}\right]
$$

avec $x = N(\mathrm{H}_2)/5\,10^{14} \, \mathrm{cm}^{-2}$ et $b_5 = b/10^5 \, \mathrm{cm}\,\mathrm{s}^{-1}$. L'écrantage affecte la formation et destruction de la molécule. Je multiplie la fonction $f_{\text {shield}}$ au champs de rayonnement $\chi$.
On obtient ainsi un nouveau taux de chauffage :

\begin{equation}
    \Gamma_{\mathrm{H}_2^\star} = n_{\mathrm{H}_2}\frac{\chi P \, f_{\text {shield}}}{1 + \frac{A_{\text{eff}}+  f_{\text {shield}} \chi D_{\text{eff}}}{\gamma_{\text{eff}} \, n}} \Delta E \operatorname{erg} \mathrm{cm}^{-3} \mathrm{s}^{-1}
\end{equation}



On fait de même pour le taux de refroidissement.

\begin{figure}[h!]
    \centering
    \includegraphics[width = 0.4\textwidth]{figure/H2/pomp/fshield.png}
    \caption{Fonction de shielding $f_\mathrm{shield}$ de $\mathrm{H}_2$ en fonction de la colonne densité $N(\mathrm{H}_2)$}
    \label{fig:H2:fshield}
\end{figure}


\subsubsection{Extinction par les grains}

Le champs de rayonnement calculé par le code ne prend pas en compte l'extinction par les grains. Approximation FGK. Je corrige le champs de rayonnement par $e^{-\tau_d}$ où $\tau_d = N_\mathrm{H}\sigma_d$ donné par \cite{SternbergLePetit2014} Eq 20. Ainsi, 

$$\chi^{'} = e^{-\tau_d
}\, f_{\mathrm{shield}}\, \chi$$


\subsection{Comparaison avec le code PDR de Meudon}

On récupère du code le taux de refroidissement $\Lambda_{\mathrm{PDR}}$ par la molécule $\mathrm{H}_2$ qui peut être positif ou négatif. Le taux prend en compte du chauffage par desexcitation collisionnelle et du refroidissement ro-vibrationelle (émission). Il ne prend pas en compte de la photodissociation (qui chauffe). On veut étudier le chauffage, on appelle $\Gamma_{\mathrm{PDR}}$ la partie négative du taux ($\Lambda_{\mathrm{PDR}} < 0$) et on le compare aux de chauffage nets.

\begin{equation}
    \begin{split}
        \Gamma_{\mathrm{Rollig} \, \mathrm{net}} &= \Gamma_{\mathrm{H}_2^\star} + \Lambda_{\mathrm{H}_2} \\ 
        \Gamma_{\mathrm{BS} \, \mathrm{net}} &=\Gamma_{\mathrm{H}_2 \, \mathrm{pump}} +  \Gamma_{\mathrm{H}_2 \, \mathrm{pd}} 
    \end{split}
\end{equation}

La figure \ref{fig:H2:GammaPDR} compare les taux de chauffages nets utilisant différentes prescriptions à celui calculé dans le code (en noir). Le chauffage calculé par Rollïg et de Bialy&Sternberg ont la même intensité en bord de nuage où la désexcitation collisionnelle est prédominante. 

L'approximation FGK est une méthode qui calcule le spectre du champs de rayonnements à travers le nuage qui prend en compte l'absorption dans le continuum et le carbone. Il prend également en compte l'écrantage de la molécule $\mathrm{H}_2$ (figure \ref{fig:H2:fgk} \cite{FGK}). 

\begin{figure}[h!]
    \centering
    \includegraphics[width = 0.8\textwidth]{figure/H2/fgk.png}
    \caption{Densité d'énergie au sein d'un nuage à un $A_\mathrm{V}=0.5$. Les niveaux du $\mathrm{H}$ et de $\mathrm{H}_2$ absorbent certains photons sur certaines raies. A mesure que l'on s'enfonce dans le nuage beaucoup de matière se trouve sur la ligne de visée et les raies d'absroptions s'élargissent (voir optiquement épaisse Draine) \cite{FGK}}
    \label{fig:H2:fgk}
\end{figure}


\subsection{Prescription de Glover et Janev}

Dans la version \uncinq les taux de dissociation collisionelle de $\mathrm{H}_2$ via 

\begin{equation}
    \begin{array}{lcccccccl}
        \mathrm{H} & + & \mathrm{H}_2   & \rightarrow &\mathrm{H}  & + & \mathrm{H} & + & \mathrm{H} \\
        \mathrm{H}_2  & + & \mathrm{H}_2  & \rightarrow & \mathrm{H} & + &\mathrm{H}_2  & + & \mathrm{H} \\
    \end{array}
\end{equation}

sont calculé selon Janev (2008) (retrouver citation). On utilise une nouvelle version du code (\unsept) qui calcule aussi les taux de dissociation selon Janev : rien ne bouge. Les raies du $\mathrm{H}_2$ et du $\mathrm{CO}$ sont les mêmes et le profils de température aussi. On utilise dorénavant pour cette étude la version \unsept. On compare la prescription de Glover à celle de Janev et on remarque plusieurs choses. Tout d'abord les raies d'émissions de $\mathrm{H}_2$ et $\mathrm{CO}$ sont augmentées (\autoref{figu:H2:..}). De plus le profil de température avec la nouvelle prescription (Glover) est modifié un tout petit peu au au bord (+100K) et un peu à l'entrée du nuage moléculaire (+400K) (\autoref{fig:H2:JanevGlover:emiss}). L'augmentation de la température à l'entrée du nuage moléculaire ($A_\mathrm{V} = 0.8$) provient du chauffage par exothermicité des réactions chimiques qui devient majeure (jusqu'à $50\%$ du chauffage total). Janev a tendance à surestimer les taux de dissociation qui sont toutes deux des réactions endothermiques et qui ont des efficacités de refroidissement les plus importantes. Les taux calculé par Glover réduisent leur refroidissement globale sur le nuage ce qui le chauffe. \newline 


\begin{figure}[h!]
    \centering
    \begin{subfigure}[t]{0.45\textwidth} % "0.45" donne ici la largeur de l'image
        \centering \includegraphics[trim = {0 0 0 1.5cm},clip,width=1\textwidth]{figure/H2/JanevGlover/I_comp_CO.png}
        \caption{Spectre $\mathrm{H}_2$}
    \end{subfigure}
    ~ 
    \begin{subfigure}[t]{0.45\textwidth}
        \centering \includegraphics[trim = {0 0 0 1.5cm},clip,width=1\textwidth]{figure/H2/JanevGlover/nT_comp_CO.png}
        \caption{Profil de densité et température de $\mathrm{H}_2$}
    \end{subfigure}

    \centering
    \begin{subfigure}[t]{0.45\textwidth} % "0.45" donne ici la largeur de l'image
        \centering \includegraphics[trim = {0 0 0 1.5cm},clip,width=1\textwidth]{figure/H2/JanevGlover/I_comp_H2.png}
        \caption{Spectre de $\mathrm{CO}$}
    \end{subfigure}
    ~ 
    \begin{subfigure}[t]{0.45\textwidth}
        \centering \includegraphics[trim = {0 0 0 1.5cm},clip,width=1\textwidth]{figure/H2/JanevGlover/nT_comp_H2.png}
        \caption{Profil de densité et température de $\mathrm{CO}$}
    \end{subfigure}
    \caption{Impact des prescriptions de Janev et Glover sur les raies d'émissions des traceurs $\mathrm{H}_2$ et $\mathrm{CO}$}
    \label{fig:H2:JanevGlover:emiss}
\end{figure}


Néanmoins on connaît mal la proportion effective qui chauffe le gaz par exothermicité alors que l'on compte sur elle pour exciter nos raies. Il faut reprendre l'étude sur l'exothermicité des réactions chimiques et jouer sur cette prescription. \newline 

On cherche à visualiser l'impact des nouveaux calculs des niveaux de $\mathrm{H}_2$ sur les raies. On l'a calculé dans le cas de Janev et Glover mais l'on montre seulement la prescription de Glover qui est la plus importante (\autoref{fig:H2:GloverBossion:emiss})

On se rend compte que l'impact est minime alors que le travail pour calculer ces niveaux est lourd. Au moins on sait que connaître précisément les niveaux de $\mathrm{H}_2$ n'est pas décisif dans l'interprétation des spectres d'émissions. Il faut explorer d'autres possibilités. 

\begin{figure}[h!]
    \centering
    \begin{subfigure}[t]{0.45\textwidth} % "0.45" donne ici la largeur de l'image
        \centering \includegraphics[trim = {0 0 0 1.5cm},clip,width=1\textwidth]{figure/H2/GloverBossion/I_comp_CO.png}
        \caption{Spectre $\mathrm{H}_2$}
    \end{subfigure}
    ~ 
    \begin{subfigure}[t]{0.45\textwidth}
        \centering \includegraphics[trim = {0 0 0 1.5cm},clip,width=1\textwidth]{figure/H2/GloverBossion/nT_comp_CO.png}
        \caption{Profil de densité et température de $\mathrm{H}_2$}
    \end{subfigure}

    \centering
    \begin{subfigure}[t]{0.45\textwidth} % "0.45" donne ici la largeur de l'image
        \centering \includegraphics[trim = {0 0 0 1.5cm},clip,width=1\textwidth]{figure/H2/GloverBossion/I_comp_H2.png}
        \caption{Spectre de $\mathrm{CO}$}
    \end{subfigure}
    ~ 
    \begin{subfigure}[t]{0.45\textwidth}
        \centering \includegraphics[trim = {0 0 0 1.5cm},clip,width=1\textwidth]{figure/H2/GloverBossion/nT_comp_H2.png}
        \caption{Profil de densité et température de $\mathrm{CO}$}
    \end{subfigure}
    \caption{Impact des calculs semi-classiques (Bossion) avec la prescription de Glover sur les raies d'émissions des traceurs $\mathrm{H}_2$ et $\mathrm{CO}$}
    \label{fig:H2:GloverBossion:emiss}
\end{figure}


\subsection{Quand le chauffage par $\mathrm{H}_2$ prédomine ?}

On trace sur la figure ... le type de chauffage prédominant en bord de nuage. En comparant à la version \uncinq sans le chlore (Janev) on se rend compte que la zone concerné par le chauffage par $\mathrm{H}_2$ est plus large et joue pour des région à plus fort champs de rayonnement. \newline

\begin{figure}[h!]
    \centering
    \includegraphics[width = 0.6\textwidth]{figure/H2/mapGloverBossion/mapGmax.png}
    \caption{}
    \label{fig:H2:mapGloverBossion:Gmax}
\end{figure}

On trace le profil de température de quelques modèles (\autoref{fig:H2:mapGloverBossion:profilTx}). 
\begin{itemize}
    \item On voit (d) que la température au bord du nuage augmente pour les fort champs de rayonnement et que cette tendance est moins vraie (c) si le chauffage par $\mathrm{H}_2$ cesse de prédominer en bord de nuage. Il vient un $\chi$ où la température augmente de manière raide à l'entrée du nuage moléculaire (a). Je présume que c'est la recombinaison sur les grains qui est accéléré par le début de formation de $\mathrm{H}_2$ mais A VERIFIER. Comparer les $\Gamma$ et $\Lambda$ en fonction de la profondeur du nuage pour quelques modèles qui nous intéresse.
    \item Si l'on s'enfonce en $\chi$ (b) le chauffage par $\mathrm{H}_2$ commence à prédominer en bord de nuage et on voit qu'il efface l'augmentation de température en bord de nuage moléculaire. Cela se voit aussi en comarant (a) et (c).
    \item A forte densité (d) il apparaît un plateau après la transition $\mathrm{H}/\mathrm{H}_2$ : est ce que cela est du à l'effet photoélectrique ? pourrait il exciter d'avantage les raies du H2 ? On a vu avant qu'elle pouvait être du au chauffage par réactions chimiques qui devenait important car Glover tue la dissociation du H2. Et si l'on tracait ce T à cet endroit du nuage dans l'espace des paramètres. Peut être que l'on verrait aussi un bistabilité ? (première fois ou le gradient de T est minimale après la transition)
    \item La température de la transition ne change pas d'un poil avec $\chi$ mais bouge avec la densité. Un fort champ de rayonnement ne fera que déplacer l'AV mais pas la température. Se comprend un peu car $\Gamma \propto n^2$ et $\Lambda \propto n $ mais pas toujours $\propto \chi$.
\end{itemize}

\begin{figure}[h!]
    \centering
    \begin{subfigure}[t]{0.45\textwidth} % "0.45" donne ici la largeur de l'image
        \centering \includegraphics[trim = {0 0 0 0},clip,width=1\textwidth]{figure/H2/mapGloverBossion/H2_n_1p7_bossion_d3p5r2p5_d3p5r3p5_d3p5r4p5.png} 
        \caption{}
    \end{subfigure}
    ~ 
    \begin{subfigure}[t]{0.45\textwidth}
        \centering \includegraphics[trim = {0 0 0 0},clip,width=1\textwidth]{figure/H2/mapGloverBossion/H2_n_1p7_bossion_d3p5r4p5_d4p5r4p5_d5p5r4p5.png}
        \caption{}
    \end{subfigure}

    \centering
    \begin{subfigure}[t]{0.45\textwidth} % "0.45" donne ici la largeur de l'image
        \centering \includegraphics[trim = {0 0 0 0},clip,width=1\textwidth]{figure/H2/mapGloverBossion/H2_n_1p7_bossion_d4p5r2p5_d4p5r3p5_d4p5r4p5.png} 
        \caption{}
    \end{subfigure}
    ~ 
    \begin{subfigure}[t]{0.45\textwidth}
        \centering \includegraphics[trim = {0 0 0 0},clip,width=1\textwidth]{figure/H2/mapGloverBossion/H2_n_1p7_bossion_d5p5r2p5_d5p5r3p5_d5p5r4p5.png} 
        \caption{}
    \end{subfigure}
    \caption{Profil de températures pour différent modèles de la grille (Glover + Bossion). La croix indique la température de la transition $\mathrm{H}/\mathrm{H}_2$ : $n(\mathrm{H})=2n(\mathrm{H}_2)$.}
    \label{fig:H2:mapGloverBossion:profilTx}
\end{figure}

On observe également la température de la transition $\mathrm{H}/\mathrm{H}_2$ dans l'espace des paramètres (\autoref{fig:H2:mapGloverBossion:mapTHH2}). La température ne dépend toujours pas de l'intensité du champs de rayonnements. On voit que l'on arrive à dépasser le plateau des (600K) que Franck avait avec Janev. La dissociation étant moins importante on peut avoir du gaz plus chaud. Il sera plus facile de comprendre si c'est uniquement grâce à Bossion ou Glover. \newline 

\begin{figure}[th!]
    \centering
    \begin{subfigure}[t]{0.45\textwidth} % "0.45" donne ici la largeur de l'image
        \centering \includegraphics[trim = {0 0 0 0},clip,width=1\textwidth]{figure/H2/mapGlovernoBossion/HH2_T_Franck.png}
        \caption{}
    \end{subfigure}
    ~ 
    \begin{subfigure}[t]{0.45\textwidth}
        \centering \includegraphics[trim = {0 0 0 0},clip,width=1\textwidth]{figure/H2/mapGloverBossion/HH2_T_Franck.png}
        \caption{}
    \end{subfigure}
    \caption{Carte de températures de la transition $\mathrm{H}/\mathrm{H}_2$ : $n(\mathrm{H})=2n(\mathrm{H}_2)$ avec ou sans les niveaux de Bossion (toujours Glover)}
    \label{fig:H2:mapGloverBossion:mapTHH2}
\end{figure}

\underline{Conclusion partielle :} il reste à s'assurer si la zone concernée est du à la nouvelle prescription de Glover ou aux calculs semi classique des niveaux de $\mathrm{H}_2$. Sinon, on voit que le chauffage par $\mathrm{H}_2$ concerne les régions denses et impactent principalement les bords atomiques. Or c'est si l'on chauffait l'entrée du nuage que l'on pourrait espérer observer les changements. Il faudrait regarder les diagrammes d'excitations de quelques espèces comme $\mathrm{H}_2$ ou $\mathrm{CO}$. On se pose également une question sur l'impact de la raideur de la transition sur la température (\autoref{fig:H2:mapGloverBossion:smooth}) : dans certains cas, la transition se fait pour des gaz encore chaud ce qui signifie que l'on pourrait voir des raies plus excitées. Dans quelles conditions ces phénomènes ont ils lieux ? Est ce que cela a un impact sur les raies ? (Sternberg 2014) Enfin on n'a pas observé les instabilités que pouvait provoquer le $\mathrm{H}_2$. Tracer quelques courbes de chauffages et refroidissements pour différent modèles (là le gradient de Tmax) est maximale par exemple. On observait également mieux l'instabilité sur des modèles isobares : faire une petite grille. Pour avancer sur cette histoire de chauffage par réaction chimiqie : tracer $\max_{AV} \Gamma_{ch}/\Gamma_{tot}$ dans l'espace des paramètres et le $argmax_AV  \Gamma_{ch}/\Gamma_{tot}/AV_{HH2}$  qui nous permettra au moins de voir ou le chauffag epar réactions chimiques peut jouer beaucoup. 

\begin{figure}[th!]
        \centering \includegraphics[trim = {0 0 0 0},clip,width=0.5\textwidth]{figure/H2/mapGloverBossion/H2_n_1p7_bossion_d3p5r2p5_d4p5r2p5_d5p5r2p5.png}
        \caption{Profils de températures de différents modèles de la grille (Glover + Bossion)}
        \label{fig:H2:mapGloverBossion:smooth}
\end{figure}




\clearpage
\section{Explication du code python}
J'ai écris pendant mon stage des scripts python pour manipuler le code PDR ainsi que pour traiter les données de sorties. De manière générale, tout est rangé dans le dossier \texttt{./Result\_test} d'où je travaillais. Certaines fonctions faisaient appels à des parties du code PDR comme l'extracteur ou les fichiers des raies des traceurs. J'utilisais pour ma part cette version du code :\texttt{./PDR155stag\_SVN} qui était rangée au même niveau que \texttt{./Result\_test}. J'expose ici l'arborescence de mes fichiers. 

\subsection{Result\_Test/}

\textit{repository des .in : pdrin} 

\textit{repository des scripts : analysePDR} \newline 


C'est le dossier contenant les \textit{.txt} des modèles ainsi que les scripts d'analyse. \newline 

Les dossiers commençant par \texttt{OrbarJ18\_BestMod\_*}, \texttt{PDR155\_*} et \texttt{Neufire\_*} (contraction astucieuse de Neufeld \& Wolfire) sont des modèles PDR résolus par le code. Ces dossiers comportent un dossier \texttt{brupt/} qui renferment des fichiers HDF5, \textit{.bin}, \textit{.stats} voire des \textit{.res} ; un dossier \texttt{data/} qui contient les extractions des HDF5 (via l'extraceteur) ; et un dossier \texttt{figure/} qui contient les graphiques construits à partir des données de ces modèles (certains comparent plusieurs modèles à la fois et sont forcément rangés dans l'un des dossiers des modèles). \newline 

En construisant mes graphiques, j'essayais de former des fonctions génériques qui permettent de lire les fichiers, classer les tableaux et afficher des plots. Elles sont regroupées dans \texttt{plot\_hdf5.py}, \texttt{plot\_res.py} et \texttt{plotI.py}. Certaines fonctions sont devenues obsolètes. \newline

Le dossier \texttt{traceurs/} renferme un script qui compare les raies des traceurs qui sont rangées dans les fichiers \textit{.stats}. Ce script compare les \textit{.stats} de deux modèles puis écrit un tableau qui classe les rapports d'intensités des raies. En prenant le $\log_{10}$ du rapport des intensité, on obtient un score. Les rapports de raies les plus positifs et les plus négatifs sont les plus différentes entre les deux modèles . On peut ainsi comparer deux modèles PDR mais aussi que deux couples deux à deux. \newline 

Les scripts \texttt{soutenance.py} et \texttt{ED127.py} contiennent des fonctions que j'ai spécialement adaptée pour construire les graphes de l'oral de l'ED127 et de ma soutenance de master. \newline

Je vais maintenant présenter les scripts d'analyse qui sont simplement des appels à mes fonctions qui sont rangés dans \texttt{plot\_hdf5.py}, \texttt{plot\_res.py} et \texttt{plotI.py}. Historiquement, je travaillais au début du stage avec les modèles PDR du type \texttt{OrbarJ18\_BestMod\_*}, \texttt{PDR155\_*} et \texttt{Neufire\_*}. Je faisais appel à mes fonctions depuis un terminal et presque aucune trace était gardée. Puis lorsque j'ai commencé à utiliser les grilles de modèles, j'ai créé des "scripts d'analyse" qui rassemble les appels. Les modèles que j'utilisais étaient des modèles PDR de la grille et ils étaient rangés dans un des dossiers de \texttt{.Result\_test/GridSingle/}.

De plus, pour faire appel à mes modèles dans mes scripts j'utilise des objets qui renferme les noms complets de mes objets. Ils sont regroupés \texttt{config.py}. Ainsi, \textit{compareCl} est un groupe de modèles PDR utilisé pour étudier l'impact du chlore. Il comporte des attributs faisant référence aux noms ("Cl155\_n\_d5p0r4p5") ainsi qu'à son fichier \textit{.res}. Je garde toujours la structure \texttt{/brupt}, \texttt{/data} et \texttt{/figure} pour un modèle. Voici une description des scripts d'analyse :
\begin{itemize}
    \item \texttt{analysis46} compare l'impact du nouveau calcul des taux de réaction qui prend en compte l'énergie interne de $\mathrm{H}_2$, sur les raies des traceurs. On s'intéresse surtout au CO. Il ne faut pas oublier le png qui rassemble tous les réseaux de formations des molécules que j'ai pu étudier (\texttt{type46.png}).
    \item \texttt{analysisBosseChem} étudie l'augmentation locale de température au niveau de la transition $\mathrm{H}/\mathrm{H}_2$ au passage de la prescription de Janev à celle de Glover. Les diagrammes d'intensités de $\mathrm{H}_2$ et de $\mathrm{CO}$ et les courbes de chauffage et de refroidissement au niveau de la bosse sont tracés. Les profils de densités $\mathrm{H}$, $\mathrm{H}_2$ ainsi que $\mathrm{C}^+$, $\mathrm{C}$ et $\mathrm{CO}$ sont créés. Certains graphes ont été reproduit pour un modèle isobare type barre d'Orion. J'en ai également profité pour voir l'impact de Glover sur le chauffage au bord du nuage.
    \item \texttt{analysisCl} trace les diagrammes d'intensités de raies de nombreux traceurs (moléculaires ou atomiques) ainsi que des courbes de chauffage et de refroidissement pour quelques modèles de la grille. Je compare également les courbes de chauffage et de refroidissement avec celles du modèle analytique du chlore (\texttt{modelCl}).
    \item \texttt{modelCl} copie simple du modèle analytique du chlore que j'utilise pour les graphes de \texttt{analysisCl}. 
    \item \texttt{analysisGradT} : début de recherche de l'instabilité par $\mathrm{H}_2$ dans les bords atomiques des régions peu illuminées. 
    \item \texttt{analysispicT} étudie l'emballement de l'effet photoélectrique à la fin de la zone atomique pour des PDR fortement illuminées. Je compare dans un premier temps les courbes de chauffage et de refroidissement en la zone du nuage. Puis j'observe les profils de densités en fonction de la températures pour les deux mêmes modèles. 
    \item \texttt{modelC} est un premier modèle chimique expliquant l'emballement de l'effet photoéléctrique de \texttt{analysispicT}.
    \item \texttt{analysisSaddle} construit simplement un graphique pour essayer de comprendre pourquoi y a t-il un point selle sur les cartes de température des bords atomique des nuages.
    \item \texttt{analysisRes} décortique le \textit{.res}. Il reste assez brouillon. Il a surtout servi à construire la fonction \texttt{sort\_res} de \texttt{plot\_res.py} qui refait les totaux des colonnes du \textit{.res}.
\end{itemize}

Ne prenez pas en compte les fichiers \textit{.xxx.py}. 

\subsection{Grille de modèles}

\textit{repository sur gitlab : grilles} \newline

J'ai construit des grilles de modèles PDR que j'ai fait résoudre par différentes versions du code. Leurs noms signifient :
\begin{itemize}
    \item \texttt{noCl155\_n} : modèles à densité constante résolus par la version \uncinq du code PDR (+ prescription de Janev). 
    \item \texttt{noCl155\_P} : modèles à pression constante résolus par la version \uncinq du code PDR (+ prescription de Janev). Pas réellement utilisé.
    \item \texttt{Cl155\_n} : modèles à densité constante résolus par la version \uncinq du code PDR (+ prescription de Janev).  Ces modèles contiennent du chlore dans leurs réseaux chimiques.
    \item \texttt{Cl155\_P} : modèles à pression constante résolus par la version \uncinq du code PDR (+ prescription de Janev). Pas réellement utilisé. Ces modèles contiennent du chlore dans leurs réseaux chimiques.
    \item \texttt{H2\_n\_1p7\_nobossion} modèles à densités constantes résolus par la version \unsept du code PDR (+ prescription de Glover).
    \item \texttt{H2\_P\_1p7\_nobossion} modèles à pression constantes résolus par la version \unsept du code PDR (+ prescription de Glover). Pas réellement utilisé.
    \item \texttt{H2\_n\_1p7\_bossion} modèles à densités constantes résolus par la version \unsept du code PDR (+ prescription de Glover). Ces modèles utilisent les nouveaux taux de collisions du $\mathrm{H}_2$ calculé par Bossion.
    \item \texttt{H2\_P\_1p7\_bossion} modèles à pression constantes résolus par la version \unsept du code PDR (+ prescription de Glover). Pas réellement utilisé. Ces modèles utilisent les nouveaux taux de collisions du $\mathrm{H}_2$ calculé par Bossion.
\end{itemize}

Une grille comporte quatre dossiers ainsi qu'une méta. Elles sont rangés dans ma session sur tycho chez \texttt{/data/Grid}. Il y a 
\begin{itemize}
    \item le dossier du code qui est utilisé ;
    \item \texttt{/grid\_nodes} qui contient tous les \textit{.in} des modèles ;
    \item \texttt{/data} qui contient les \textit{.txt} qui sont extraits des HDF5 résultant des calculs ;
    \item et \texttt{/processed} contient les fichiers \textit{.txt} donnant sous forme d'un tableau rangé par ligne des grandeurs comme la température au bord ou l'intensité d'une raie. 
\end{itemize}

Pour créer les cartes à partir de ces fichiers, je fais appel à mes scripts rangé dans \texttt{Result\_test/Grid/}. Les dossiers de mes grilles contiennent un dossier pour ranger les figures (\texttt{/figure}) et un dossier pour ranger les données (\texttt{/processed}).
\begin{itemize}
    \item \texttt{plotT.py} construit à partir des tableaux, les cartes de températures, de chauffage maximal etc... 
    \item \texttt{plotI.py} construit les cartes d'intensités des raies. 
    \item \texttt{run.py} rassemble les appels des fonctions. 
\end{itemize}

\subsection{Modèle analytique du chlore}

\textit{repository sur gitlab : modelCl} \newline

Ce modèle est construit indépendamment du reste (\texttt{.Result\_test/ModelCl}). Il ne fait pas appel aux fonctions regroupées dans \texttt{plot\_hdf5.py}, \texttt{plot\_res.py} et \texttt{plotI.py} bien que quelles unes y soient copiées. Le coeur du modèle est écrit dans \texttt{chth.py} qui contient les calculs des coefficients de réactions, des densités et du bilan thermique. A côté, \texttt{testchem.py} est juste un bac à sable qui sert à comparer les taux de réactions utilisés dans le modèle avec ceux du code PDR. \newline

Les scripts \texttt{exploration\_T.py} et \texttt{exporation\_e.py} contiennent les fonctions qui construisent les maps de densités d'électrons et de températures et testent la robustesse du modèle. \texttt{config.py} est gère plusieurs paramètres du modèles comme la prise en compte des PAH, de la recombinaison du $\mathrm{H}^+$ sur les grains ou certains processus thermiques. D'autres fonctions y sont rangées pour manipuler les fichiers de données rangés dans \texttt{/data/}. Ils étaient là pour comparer le modèle analytique au code PDR. \newline

Le fichier \texttt{run.py} contient tous les appels des fonctions de calculs qui sont rangées dans \texttt{exploration\_T.py} et \texttt{exporation\_e.py}. On trouve en fin de script des appels à des fonctions permettant de ploter des diagrammes d'états (P,n) ou (T,P) mais cela n'a finalement pas été exploité.\newline

Enfin, l'ensemble des références utilisées par le modèle analytique sont rangées dans \texttt{/bib/}.

\subsection{Modèle du chauffage par $\mathrm{H}_2$}

\textit{repository sur gitlab : modelH2} \newline


C'est un début d'étude menée sur le chauffage par pompage UV par la molécule $\mathrm{H}_2$. Il est rangé strictement de la même manière que le modèle analytique du chlore. \newline 

On y compare le code PDR à des formules tirées de \cite{Rollig2005} et \cite{BialySternberg_2019}. Les expressions analytiques représentaient mal le chauffage lorsque l'on quitte la zone atomique du nuage. On avait commencé à étudier l'impact du calcul du transfert de rayonnement sur les profils de chauffage. Un fichier readme rassemble les formules que l'on a utilisées.



\setcounter{secnumdepth}{-1}
\clearpage
\part{Conclusion}

Les observations ne cessent de faire des découvertes surprenantes qui remettent en question notre conception des PDR. Herschel a par exemple observé du $\mathrm{CO}$ excité dans des régions plus chaudes que celles calculées dans nos modèles ce qui nous pousse à revoir les processus thermiques qui interviennent au sein des PDR. La découverte de la bistabilité induite par le chlore en bord de région atomique est un phénomène nouveau. Bien que nous n'ayons pas de traceurs atomiques pouvant déceler l'existence d'une solution chaude, les changements de volumes et de températures dues à l'instabilité peuvent avoir des conséquences dynamiques intéressantes qu'il faudrait étudier (code Hydra \cite{Bron2018}). De plus, la mise en lumière d'une nouvelle voie de formation principale du $\mathrm{CO}$ par la molécule $\mathrm{OH}$ montre sur les diagrammes d'intensité des raies l'importance d'avoir un réseau chimique à jour en terme de physique. Par ailleurs, l'utilisation des nouveaux taux de destruction du $\mathrm{H}_2$ prouve que le chauffage par les réactions chimiques, qui est généralement négligé dans les modèles de PDR, est déterminant pour prédire des diagrammes d'intensité du $\mathrm{CO}$ en accord avec les observations. \newline 

Par manque de temps, nous n'avons pas pu mesurer entièrement l'impact du calcul des nouveaux taux de collisions de $\mathrm{H}_2$. On a néanmoins décelé une nouvelle instabilité à la fin de la zone atomique du nuage qu'il faudrait approfondir. La plupart des résultats obtenus feront l'objet d'une publication. Ces travaux ont montré que l'étude des processus thermiques au sein du nuage ouvrait de nombreuses pistes pour la compréhension des PDR. Par la suite, il sera également intéressant de revisiter le chauffage par les rayons cosmiques ou le refroidissement par le couplage gaz-grains que l'on estime trop simplement. 

\clearpage
\bibliographystyle{apalike}
\bibliography{bib}

\clearpage
\setcounter{secnumdepth}{4}
\begin{appendices}

\section{Impact du chlore} 
\label{appendix:chlore}

\begin{figure}[!htbp]
    \centering
    \begin{subfigure}[t]{0.49\textwidth} % "0.49" donne ici la largeur de l'image
        \centering \includegraphics[trim = {0 0 0 0},clip,width=1\textwidth]{figure/Cl/gridModelEmiss/nT_comp_Np.pdf}
        \caption{$\mathrm{N}^+$}
    \end{subfigure}
    ~ 
   \begin{subfigure}[t]{0.49\textwidth} % "0.49" donne ici la largeur de l'image
        \centering \includegraphics[trim = {0 0 0 0},clip,width=1\textwidth]{figure/Cl/gridModelEmiss/nT_comp_N.pdf}
        \caption{$\mathrm{N}$}
    \end{subfigure}
    
    \begin{subfigure}[t]{0.49\textwidth} % "0.49" donne ici la largeur de l'image
        \centering \includegraphics[trim = {0 0 0 0},clip,width=1\textwidth]{figure/Cl/gridModelEmiss/nT_comp_S.pdf}
        \caption{$\mathrm{S}$}
    \end{subfigure}
    ~
    \begin{subfigure}[t]{0.49\textwidth} % "0.49" donne ici la largeur de l'image
        \centering \includegraphics[trim = {0 0 0 0},clip,width=1\textwidth]{figure/Cl/gridModelEmiss/nT_comp_Si.pdf}
        \caption{$\mathrm{Si}$}
    \end{subfigure}
    
    \caption{(ANNEXE) Profils de densité des traceurs impactés par l'ajout du chlore.}
    \begin{minipage}{\textwidth}
    Il est représenté en trait plein les profils du modèle avec le chlore et en trait pointillé le modèle ne contenant pas de chlore. La température est en rouge et la densité en noir.
    \end{minipage}
    \label{fig:Cl:gridModelEmiss:nT:yes}
\end{figure}

\begin{figure}[!htbp]
    \centering
    \begin{subfigure}[t]{0.49\textwidth} % "0.49" donne ici la largeur de l'image
        \centering \includegraphics[trim = {0 0 0 0},clip,width=1\textwidth]{figure/Cl/gridModelEmiss/nT_comp_CS.pdf}
        \caption{$\mathrm{CS}$}
    \end{subfigure}
    ~ 
   \begin{subfigure}[t]{0.49\textwidth} % "0.49" donne ici la largeur de l'image
        \centering \includegraphics[trim = {0 0 0 0},clip,width=1\textwidth]{figure/Cl/gridModelEmiss/nT_comp_H2O.pdf}
        \caption{$\mathrm{H}_2\mathrm{O}$}
    \end{subfigure}
    
    \begin{subfigure}[t]{0.49\textwidth} % "0.49" donne ici la largeur de l'image
        \centering \includegraphics[trim = {0 0 0 0},clip,width=1\textwidth]{figure/Cl/gridModelEmiss/nT_comp_H2.pdf}
        \caption{$\mathrm{H}_2$}
    \end{subfigure}
    ~ 
    \begin{subfigure}[t]{0.49\textwidth} % "0.49" donne ici la largeur de l'image
        \centering \includegraphics[trim = {0 0 0 0},clip,width=1\textwidth]{figure/Cl/gridModelEmiss/nT_comp_CO.pdf}
        \caption{$\mathrm{CO}$}
    \end{subfigure}
    
    \caption{((ANNEXE) Profils de densité des traceurs impactés par l'ajout du chlore. Mêmes conventions que pour la figure \ref{fig:Cl:gridModelEmiss:nT:yes}.}
    \label{fig:Cl:gridModelEmiss:nT:no}
\end{figure}


\begin{figure}[!h]
    \centering
    \begin{subfigure}[t]{0.49\textwidth} % "0.49" donne ici la largeur de l'image
        \centering \includegraphics[trim = {0 0 0 1cm},clip,width=1\textwidth]{figure/Cl/gridModelEmiss/I_comp_CS.pdf}
        \caption{$\mathrm{CS}$}
    \end{subfigure}
    ~ 
   \begin{subfigure}[t]{0.49\textwidth} % "0.49" donne ici la largeur de l'image
        \centering \includegraphics[trim = {0 0 0 1cm},clip,width=1\textwidth]{figure/Cl/gridModelEmiss/I_comp_H2O.pdf}
        \caption{$\mathrm{H}_2\mathrm{O}$}
    \end{subfigure}
    
    \begin{subfigure}[t]{0.49\textwidth} % "0.49" donne ici la largeur de l'image
        \centering \includegraphics[trim = {0 0 0 1cm},clip,width=1\textwidth]{figure/Cl/gridModelEmiss/I_comp_H2.pdf}
        \caption{$\mathrm{H}_2$}
    \end{subfigure}
    ~ 
    \begin{subfigure}[t]{0.49\textwidth} % "0.49" donne ici la largeur de l'image
        \centering \includegraphics[trim = {0 0 0 1cm},clip,width=1\textwidth]{figure/Cl/gridModelEmiss/I_comp_CO.pdf}
        \caption{$\mathrm{CO}$}
    \end{subfigure}
    
    \caption{(ANNEXE) Diagramme d'excitation des traceurs peu modifiés par l'ajout du chlore dans le code PDR (même convention que la figure \ref{fig:Cl:gridModelEmiss:yes}). Pour la molécule $\mathrm{CO}$, les transitions rotationnelles ont été écrites (toutes s'effectuent à $\mathrm{v}=0$).}
    \label{fig:Cl:gridModelEmiss:no}
\end{figure}


%%%%%%%%%%%%%%%%%%%%%%%%%%%%%%%%%%%%%%%%%%%%%%%%%%%%%%%%%%%%%%%%%%%%%%%%%%%%%%%%%%%%%%%%%%%%%
\clearpage
\section{$\mathrm{H}_2$ excité}
\label{appendix:type46}
\begin{figure}[!h]
    \centering \includegraphics[trim = {0 0 0 1cm},clip,width=0.8\textwidth]{figure/type46/I_comp_H2.pdf}
    \caption{Diagramme d'excitation du $\mathrm{H}_2$}
    \begin{minipage}{\textwidth}
    
    \end{minipage}
    \label{fig:type46:H2}
\end{figure}

\begin{figure}[!h]
    \centering \includegraphics[trim = {0 0 0 1cm},clip,width=0.8\textwidth]{figure/type46/I_comp_H2O.pdf}
    \caption{Diagramme d'excitation du $\mathrm{H}_2\mathrm{O}$}
    \begin{minipage}{\textwidth}
  
    \end{minipage}
    \label{fig:type46:H2O}
\end{figure}



\end{appendices}





\end{spacing}

\end{document}